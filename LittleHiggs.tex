\documentclass[aps,preprint,floatfix,nofootinbib,showpacs]{revtex4-1}
\usepackage{graphicx,color,epstopdf,amsmath,multirow,array,booktabs}
\usepackage{feyn}
\usepackage[T1]{fontenc}
\usepackage{kotex}

\newcommand{\head}[1]{\textnormal{\textbf{#1}}}
\newcommand{\normal}[1]{\multicolumn{1}{l}{#1}}
\newcommand{\ev}{{\;{\rm eV}}}
\newcommand{\kev}{{\;{\rm keV}}}
\newcommand{\mev}{{\;{\rm MeV}}}
\newcommand{\gev}{{\;{\rm GeV}}}
\newcommand{\tev}{{\;{\rm TeV}}}
\newcommand{\pb}{{\,{\rm pb}}}
\newcommand{\fb}{{\,{\rm fb}}}



\begin{document}
\title{Little Higgs model basic study}
\bigskip
\author{So Young Shim$^{1}$}
\affiliation{
$^1$DESY, Notkestrasse, Hamburg, 22607, Germany
}
\date{\today}

\begin{abstract}
With the SM, the Higgs self-coupling radiative correction could cause infinite mass of Higgs. 
It is because there is no symmetry protection for this radiative correction.
But, in Little Higgs model, introducing extra symmetries 
to ones of the SM prevent Higgs mass from divergence.
If integrating out heavy particles in Little Higgs model, 
some operators look like dim-6 operators.
Using this, I study the phenomenological part with WHIZARD with dim6 operators.\\
Little Higgs model 
\end{abstract}

\maketitle


\section{Little Higgs model}
\label{sec:introduction}
Generally Little Higgs models have different group structures 
for global and local gauge symmetries.
In this kind of models, Higgs is regarded as one of the Nambu Goldstone bosons.
They have enlarged gauge group structure from that of SM and 
break the group into the same one with SM. 
Basically these models have more gauge bosons than the SM does. 
{\it Gauge bosons come from local gauge group and Nambu Goldstone 
bosons are from global gauge symmetry breaking.}
When this breaking occurs, extra heavy gauge bosons appear as well as the SM Higgs show up. 
% Effective field theory, higher dimensional operators,
% there is only 1 5-dimensional operator 
% because of Majorana mass term.
% But, 6-dimensional operators are various. 

\section{Nambu Goldstone boson}
According to Goldstone theorem, when each symmetry breaks, a massless particle comes out. 
In the minimal Standard model, breaking $SU(2)_L \times U(1)_Y$ group 
into $U(1)_{em}$ makes a Higgs.
In that case, the number of generators of the group has changed 4 into 1. 
Then 3 massless particles, 
i.e. 3 of Nambu Goldstone bosons come out.
And, these NGBs become the longitudinal component of gauge bosons. 
Usually, we say this situation as ``NGBs are eaten by gauge bosons''.

And what I read in the ``Understanding the Standard Model'' 
(author : S.K. OH) was the other way around.
That is, when the symmetries break into $U(1)_{em}$, 
its corresponding Goldstone boson come out and 
one of them is photon, others are eaten by Gauge bosons and then, 
the left degree of freedom is just for Higgs.

% I am not sure my memory and understanding is correct. 
Now, I can see the Nambu Goldstone theorem and the book have talked the same thing.

% However, in the Little Higgs model, they consider 
% the Higgs, scalar gauge bosons as Nambu-Goldstone bosons. 
% How the group structure look like 
% before the symmetries breaking really depends on the models.



\newpage
\section{the Littlest Higgs model}
The Littlest Higgs model has $SU(5)/SO(5)$ as global symmetry and $(SU(2)\times U(1))^2$ 
as local symmetry.

% Honestly, in this part, what I can't understand is 
% the fact that the d.o.f. are not consistant.
For global symmetry, the number of d.o.f. of $SU(5)$ is $5^2 -1 = 24$ and 
that of $SO(5)$ is $\frac{5\times4}{2} = 10$. 
For local gauge group, they change from $(3+1)\times 2 = 8$ to 4. 

And of course, the group structure of ths SM should be encoded 
in the group structre before symmetry breaking.

For local gauge symmetry,  $(SU(2)\times U(1))^2$ becomes  $SU(2)\times U(1)$. 
So, there are 4 additional gauge bosons: $W_H^-, W_H^+, Z_H, A_H$

From global gauge symmetry, $SU(5)$ is broken into $SO(5)$. 14 NGB show up. 
$2\times2$ are for SM Higgs doublet, $3\times2$ 
are for scalar triplet which is in the metric breaking $SU(5)$.
And 4 of them are eaten by the heavy gauge bosons.

{\large{But, in the \it{Pseudo-Axion} paper,}} the authors have some special assumption,
which is that one $U(1)$ of $(SU(2) \times U(2))^2$ is not gauged. 
So, there is no heavy photon, $A_H$, but one NGB uneaten, $\eta$ is left .

\newpage
\section{Pseudo-Axion in the Littlest Higgs model}
Once I calculated the Higgs mass and pseudoscalar mass 
with a part of Lagrangian in the $L^2 HM$ paper. 
But, the Lagrangian was not for Higgs or scalar mass 
and I made some mistakes for the calculation.
I got some equtions which looked not very plausible 
though the results were supposed to be zero. 
(Every term should have been vanished.) 

Anyway, I start another calculation with the low-energy structure paper for $L^2 HM$. 




\newpage
\section{the Simplest Little Higgs model}
The Simplest Little Higgs model has $SU(3)_1\times SU(3)_2$ as global gauge group and 
$SU(3)\times U(1)$ as local gauge symmetry.

NOW, although I am really confused, JUST let me write down as far as I know.

\subsection{20.Feb.2016 the confusing part }
So, I think now I understand somehow.

For local gauge structure, it starts with $SU(3)\times U(1)$ 
which becomes $SU(2)\times U(1)$ after symmetry breaking.
There are 9 of gauge bosons: $W^-_H, W^+_H, Z_H, W^-, W^+, Z, B, H, \eta$.
Three of them get mass when the group structure becomes $SU(2)\times U(1)$. 
Fermions that are doublet in ths SM, also start in the form of triplets. 

There is global gauge group $SU(3)_1\times SU(3)_2$(16), which is going to be broken
to $SU(2)_L \times U(1)_Y$(4). Counting degree of freedoms, 16 d.o.f becomes 4, {\it i.e.}
12 of generators are broken. 5 of them are eaten by heavy gauge bosons; $W^-_H, W^+_H, Z_H$
4 of them will be eaten by SM higgs doublet ($h, h\dagger $)

{\large{Only one NGB should be left but why are there 3 of them left??????}}



The group's unbroken generators can be 
parameterized by 
\begin{equation}
\label{Sigma0}
\Sigma_0 = \left( \begin{array}{ccc}
 & & {\mathbf{1}}_{2 \times 2} \\
 &1 & \\
{\mathbf{1}}_{2 \times 2} & &
\end{array}\right).
\end{equation}



\newpage
\section{Pseudo-Axion in the Simplest Little Higgs model}
Pseudo-axion appears from NGB fields. Actually, it is one of the NGBs. 
In case of the SLHM, the corresponding terms live in the metrics 
which break $SU(3)_1\times SU(3)_2$. 
The pseudoscalar, $\eta$, occupy diagonal parts of the metric. 
The symmetry breaking scales for each group are denoted by $f_1, f_2$ and
for convinience, there is $f^2 = f_1^2+f_2^2$. 

OMG, I am confused again. So, is the metric where the pseudo-axion live 
generators of the group? 
Or, a part of the vacuum expectation value which break the symmetry??
Battery is weird. It is charging only until 89 percents.. 


\bigskip
\bigskip

Basically, the scalar particles' mass terms come from their potential. 
Considering Coleman-Weinberg potential, mass terms of scalar particles can be formed.

In SLHM's CW potential, they depend only on $\mu^2$ term. 
The mass of Higgs is affected by Coleman-Weinberg potential correction. 
\begin{equation}
 m^2_H = -2 (\delta m^2 + m^2_\eta)
\end{equation}
where 
\begin{equation}
 \delta m ^2 = \frac{-3}{8 \pi^2}(\lambda^2_t m^2_T Log(\frac{\Lambda^2}{m^2_T}) 
 - \frac{g^2}{4} m^2_{W_H} Log(\frac{\Lambda^2}{m^2_{W_H}}) - 
 \frac{g^2}{8}(1+t^2)m^2_{Z_H} Log(\frac{\Lambda^2}{m^2_{Z_H}}))
\end{equation}
and 
\begin{equation}
 m_\eta = \sqrt{\kappa}\mu
\end{equation}



\end{document}
% 
