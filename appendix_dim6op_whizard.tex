% %%%%%%%%%%%%%%%%%%%%%%%%%%%%%%%%%%%%%%%%%%%%%%%%%%%%%%%%%%%%%%%%%%
% Template for PhD Thesis at Hamburg University
%
% \file appendix.tex
%
% Author: Markus Ebert
%
% Copyright: (c) Markus Ebert, 2017
%    This work may be freely used, modified and redistributed.
%    Good luck with your PhD!
%
% Description:
%    Appendix.
% %%%%%%%%%%%%%%%%%%%%%%%%%%%%%%%%%%%%%%%%%%%%%%%%%%%%%%%%%%%%%%%%%%

%%%%%%%%%%%%%%%%%%%%%%%%%%%%%%%%%%%%%%%%%%%%%%%%%%%%%%%%%%%%%%%%%%%%%%%%%%%%%%%%
\chapter{Dimension-6 operators in \whizard}
\label{app:notation}
%%%%%%%%%%%%%%%%%%%%%%%%%%%%%%%%%%%%%%%%%%%%%%%%%%%%%%%%%%%%%%%%%%%%%%%%%%%%%%%%

\section{How to implement new vertices into \whizard}

% Look up whether there is the vertex which consists of the same particles. 
% If there is, you can just use it.
% If not, you should define yourself. 


In WHIZARD, the parts that should be modified for implementation 6-dimensional operators are 7 files.
In \texttt{/trunk/omega/src} directory, the files are \texttt{omegalib.nw, targets.ml, modellib\_SM.ml, couplings.mli}, and 
\texttt{colorize.mli}. \texttt{omegalib.nw} includes funtions for vetices. 
If you have new Lorentz structure, you should make and
define new vertex funtions by LSZ reduction theorem. 
\texttt{targets.ml} have a distrubution role for the vertex funtions. 
It distributes particles' Lorentz index numbers and corresponds the functions
to the Lorentz structures' name in \texttt{modellib\_SM.ml}. 
\texttt{ modellib\_SM.ml} defines vertices kinds, particles, Lorentz structures, and coupling coefficients.
In \texttt{/trunk/share/models/} directory, \texttt{ SM\_ac.ml}, 
which defines parameters such as Wilson coefficients.
And, in \texttt{ /trunk/src/models/} directory, \texttt{parameters.SM\_ac.f90}. 
In this file, you can specify the coupling coefficients how to calculate. 

\subparagraph{Steps of writing code for WHIZARD} are the below.
\begin{enumerate}
 \item \texttt{/trunk/omega/src/omegalib.nw} : define functions for Lorentz structures.
 \item \texttt{/trunk/omega/src/targets.ml} : define the name of Lorentz structures for using in modellib\_SM file and write {\it fail} messages for OVM. 
 \item \texttt{/trunk/omega/src/couplings.mli} : for LaTeX
 \item \texttt{/trunk/omega/src/colorize.ml} : informations for {\it (coupling coefficient)* (color factor)}
 \item \texttt{/trunk/omega/src/modellib\_SM.ml} : define vertices with coupling coefficients and the name of Lorentz structurers
 \item \texttt{/trunk/share/models/SM\_ac.ml} : initialize the Wilson coefficients 
 \item \texttt{/trunk/src/models/parameters.SM\_ac.f90} : specify the coupling coefficients. Note that when defining the variables in this file, 
 they shoud be in the same order in the \texttt{/trunk/share/models/SM\_ac.ml}, or some variables could have completely different values.
\end{enumerate}

\subparagraph{Make and make install}
After modifying all of these files properly, going to build directory, and type {\it \$ make} and {\it \$ make install}.
If it shows error messages, you should find where the error is and modify it.
Or, you can move to compatability test. 
\begin{table}
 \begin{tabular}{@{}l*2{>{\:}l} l<{}@{}}
%  {c|c|c|c}
%   \toprule[1.5pt]
  & \multicolumn{1}{c}{Particles} &
    \multicolumn{2}{c}{Operators}\\
%   & \normal{\head{Command}} &
%   \normal{\head{Declaration}} & \normal{}\\
%   \cmidrule(lr){2-3}\cmidrule(l){4-4}
  Triple Vertices &  $(\gamma_1, \gamma_2, H_3)$ && $\mathcal{O}_{\Phi B}$ \\
  &  $(\gamma_1, H_2, Z_3)$ && $\mathcal{O}_{DW}$, $\mathcal{O}_{DB}$, $\mathcal{O}_{D\Phi W}$, $\mathcal{O}_{D\Phi B}$, $\mathcal{O}_{\Phi B}$\\
  &  $(\gamma_1, W^{-}_2, W^{+}_3)$ && $\mathcal{O}_{DW}$, $\mathcal{O}_{D\Phi W}$, $\mathcal{O}_{D\Phi B}$, $\mathcal{O}_{W}$\\
  &  $(H_1, H_2, H_3)$ && $\mathcal{O}_{6}$, $\mathcal{O}_{\Phi}$\\
  &  $(H_1, W^{-}_2, W^{+}_3)$ && $\mathcal{O}_{DW}$, $\mathcal{O}_{D\Phi W}$\\
  &  $(H_1, Z_2, Z_3)$ && $\mathcal{O}_T$, $\mathcal{O}_{DW}$, $\mathcal{O}_{DB}$, $\mathcal{O}_{D\Phi W}$, $\mathcal{O}_{D\Phi B}$, $\mathcal{O}_{\Phi B}$\\
  &  $(W^{-}_1, W^{+}_2, Z_3)$ && $\mathcal{O}_{DW}$, $\mathcal{O}_{D\Phi W}$, $\mathcal{O}_{D\Phi B}$, $\mathcal{O}_{W}$\\
%   \bottomrule[1.5pt]
 \end{tabular}
\end{table}

% 
\begin{table}
\begin{tabular}{@{}l*2{>{\:}l} l<{}@{}}
  & \multicolumn{1}{c}{{Particles}} &
    \multicolumn{2}{c}{{Operators}}\\
%   & \normal{\head{Command}} &
%   \normal{\head{Declaration}} & \normal{}\\
%   \cmidrule(lr){2-3}\cmidrule(l){4-4}
  Quartic Vertices &  $(\gamma_1, \gamma_2, H_3, H_4)$ && $\mathcal{O}_{\Phi B}$ \\
  &  $(\gamma_1, \gamma_2, W^{-}_3, W^{+}_4)$ && $\mathcal{O}_{DW}$, $\mathcal{O}_{W}$ \\
  &  $(\gamma_1, H_2, H_3, Z_4)$ && $\mathcal{O}_{DW}$, $\mathcal{O}_{DB}$, $\mathcal{O}_{D\Phi W}$, $\mathcal{O}_{D\Phi B}$, $\mathcal{O}_{\Phi B}$\\
  &  $(\gamma_1, H_2, W^{-}_3, W^{+}_4)$ && $\mathcal{O}_{DW}$, $\mathcal{O}_{D\Phi W}$, $\mathcal{O}_{D\Phi B}$\\
  &  $(\gamma_1, W^{-}_2, W^{+}_3, Z_4)$ && $\mathcal{O}_{DW}$, $\mathcal{O}_{D\Phi W}$, $\mathcal{O}_{W}$\\
  &  $(H_1, H_2, H_3, H_4)$ && $\mathcal{O}_{6}$, $\mathcal{O}_{\Phi}$\\
  &  $(H_1, H_2, W^{-}_3, W^{+}_4)$ && $\mathcal{O}_{DW}$, $\mathcal{O}_{D\Phi W}$\\
  &  $(H_1, H_2, Z_3, Z_4)$ && $\mathcal{O}_T$, $\mathcal{O}_{DW}$, $\mathcal{O}_{DB}$, $\mathcal{O}_{D\Phi W}$, $\mathcal{O}_{D\Phi B}$, $\mathcal{O}_{\Phi B}$\\
  &  $(H_1, W^{-}_2, W^{+}_3, Z_4)$ && $\mathcal{O}_{DW}$, $\mathcal{O}_{D\Phi W}$, $\mathcal{O}_{D\Phi B}$\\
  &  $(W^{-}_1, W^{+}_2, W^{-}_3, W^{+}_4)$ && $\mathcal{O}_{DW}$, $\mathcal{O}_{D\Phi W}$, $\mathcal{O}_{W}$\\
  &  $(W^{-}_1, W^{+}_2, Z_3, Z_4)$ && $\mathcal{O}_{DW}$, $\mathcal{O}_{D\Phi W}$, $\mathcal{O}_{W}$\\
%   \bottomrule[1.5pt]
\end{tabular}
\end{table}
\section{How to use the dimension-6 operators in \whizard}

here is an example for using 6 dimensional operatros.

Model: SM-6dim
