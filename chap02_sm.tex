\chapter{The Standard Model}
% \section{}
\label{sec:bsm}

This chapter elaborate on the base

The main resources of this chapter are \mycites{Gunion:1989we}, \cite{Koks:1976zec}, \cite{Bardon:1965sd}, 
\cite{deGroot:1978feq}.

\bigskip
% \more{standard model briefly}

\section{the Higgs mechanism}
% \more{just for studying :D}
To be invariant under the gauge symmetry, the mass terms of gauge bosons are set to be zero. 
But from the experiments we know gauge bosons are massive except for photons.
As introducing a scalar particle and redefining the minimum point of its potential algebraically, 
we can give mass terms to gauge bosons. Let us have a look a bit more details of the Higgs mechanism.

Assuming simpler situation, \emph{Abelian case}, we write down a Lagrangian 
\begin{align}
 \cL ~=~ &\cL_{\rm{kin}}(\phi,A^\mu) ~- V(\phi) \nonumber\\
  =& (D^\mu \phi)^* (D^\mu \phi) - \frac{1}{4}F^{\mu\nu}F_{\mu\nu} + \mu^2\phi^*\phi - \lam(\phi^*\phi)^2
\end{align}
that is invariant under the gauge transformation;
\begin{align}
 \phi ~\to &~\phi'~=~e^{ig\chi(x)}\phi \nonumber\\
 A^\mu ~\to& A'^\mu ~=~ A^\mu - \partial^\mu \chi(x) \nonumber\\
 D^\mu ~=~& \partial^\mu + i g A^\mu
\end{align}
where $\phi$ is a complex scalar field, $A$ is a gauge boson and $F^{\mu\nu}=\partial^\mu A^\nu - \partial^\nu A^\mu$ is antisymmetric tensor.
$D^\mu$ is the covariant derivative for the invariance under its gauge. 

The potential $V(\phi)$ has its minimum at $\phi~=~\frac{v}{\sqrt{2}}=\sqrt{\frac{\mu^2}{2\lam}}$, so 
redefining the scalar field to see its minimum explicitly, 
\begin{equation}
 \phi ~=~ \frac{(v+ h(x))}{\sqrt{2}} \nonumber
\end{equation}
where $h(x)$ is a real field. Expanding the Lagrangian with this new expression, 
\begin{align}
 \cL~=~\frac{1}{2}\big[ (\partial_\mu - igA_\mu)(v+h)(\partial_\mu + igA_\mu)(v+h) \big] 
 + \frac{1}{2}\mu^2 (v+h)^2 - \frac{1}{4}\lam(v+h)^4 - \frac{1}{4} F^{\mu\nu} F_{\mu\nu} \nonumber
\end{align}

This expression give us the mass term of gauge boson $A$ as $(g^2 v^2/2)A^\mu A_\mu$, and that of the scalar boson $h$ as $\lam v^2 h^2$. 
Note that we can observe terms that are related with field strength and self-interaction, $h^3$, $h^4$, $h^2 AA$, and etc.
Another insteresting thing is the fact that $\lam$ is related to the mass of $h$ and not other constraints at all. 


% \section{Weinberg-Salam model?}

\paragraph{Higgs-self coupling; quadratic divergent terms}
\section{the Electroweak Precision Test}
% 
% One of the most successful achievement of the SM is the prediction and the measurement of SM gauge bosons. 
 
%  Although there is variation according to the renormalisation scheme, 
% Z boson mass is measured and very consistent with the value from theory. 
 
% In \mycite{Agashe:2014kda} \more{this is PDG}, Z boson mass is \unit[91]{GeV}, W boson mass is \unit[89]{GeV}?\\
% The weakmixing angle, the weak interaction coupling constant $g$ is $0.3$?.
And other values are shown in the \more{table:EWPT}. \\
%  \mycite{Burgess:1993vc} model independent ewpt constraints on NP, oblique parameters related new physics\\
 \mycite{Peskin:2017emn} Peskin TASI lecture for Electroweak Precision observables.
 
The SM has been tested from several ineterations: polarization in $\beta$-decay, Muon decay, Pion decay, 
 Neutrion deep inelastic scattering, and etc.

 
\paragraph{S,T, and U parametrisation}
\mycite{Peskin:1991sw} original peskin's oblique parameter \\

\mycite{Barbieri:2004qk}seven are the remaining coefficients which enclose possible
electroweak corrections, and are expressed in terms of the following oblique parameters\\


 
 \paragraph{Precision test}
 As a known electroweak theory, the Standard Model has predicted many observables precisely 
and proved its validity by experiments. To fullfill this purpose the important quantity is 2-point function 
which is applied to describe the SM observables as well as to relate them to the quantity 
from the extension of the SM.
% at the LHC : polarization in $\beta$-decay \cite{Koks:1976zec}, Muon decay\cite{Bardon:1965sd}, 
% Pion decay, Neutrion deep inelastic scattering\cite{deGroot:1978feq}, and etc.
% Also, the experiments of $Z$ boson resonance via $e^-e^+$ annihilation has been performed with high accuracy not only to validate the SM 
% but also to suggest guide lines to beyond the SM sector. 
% To fullfill this purpose the important quantity is 2-point function 
% which is applied to describe the SM observables as well as to relate them to the quantity 
% from the extension of the SM.

The general expression of the Lagrangian related with the \emph{oblique} radiative corrections 
invariant under $U(1)_{em}$ symmetry is 
\begin{align}
 \cL_{V^2} = -\frac{1}{2}W_3^\mu \Pi_{33}(p^2) - W_3^\mu \Pi_{3B}(p^2) B_\mu 
 -\frac{1}{2} B^\mu \Pi_{BB}(p^2)B_\mu - W_+^{\mu}\Pi_{+-}(p^2)W_{-\mu}~, 
\end{align}
where $\Pi(p^2)$ is the vacuum polarization amplitudes at the tree level. 
Since this Lagrangian is formed based on the effective formalism, new effects from new physics
can be projected into it by parameterising the additional contributions.

 Assuming $q^2$ is small, the amplitude of the vacuum polarizations can be expanded as
 \begin{align}
  \Pi(p^2) ~=~\Pi(0)+p^2 \Pi'(0) + \frac{p^4}{2} \Pi''(0) + \cO(p^6). 
 \end{align}

 12 of observables are parametetrised from this Lagrangian. 
 Some of them are used for renormalisation of SM observables 
 and others are relevant to the BSM quantities.  
 These three fix the gauge kinetic terms $g,~g'$ and the mass of $W$ boson:
\begin{align}
 \Pi_{+-}'(0)~=~\Pi_{BB}'(0)~=~1, \nonumber\\
 \Pi_{+-}(0)~=~ -m_W^2 .
\end{align}

And from the fact that the photon is massless two more constraints are given:
\begin{align}
 \Pi_{\gamma\gamma}(0) = \Pi_{Z\gamma}(0) = 0 \nonumber
\end{align}

The remaining coefficients are expressed in terms of the oblique parameters
\mycite{Burgess:1993vc}
\begin{align}
\hat{S}=\frac{g}{g'}\Pi_{3B}'(0), ~~ \hat{T}=\frac{\Pi_{33}(0)-\Pi_{+-}(0)}{m_W^2},~~ 
\hat{U}=\Pi_{+-}'(0)-\Pi_{33}'(0),
\end{align}
\begin{align}
&V= \frac{m_W^2}{2}( \Pi_{33}''(0)-\Pi_{+-}''(0) )~,~~~~&X= \frac{m_W^2}{2} \Pi_{3B}''(0)~~~\nonumber\\
&Y= \frac{m_W^2}{2} \Pi_{BB}''(0) ~,~~ ~&W= \frac{m_W^2}{2} \Pi_{33}''(0)~~~ 
\end{align}
\more{MORE explanation for this}

The parameters $\hat{S}, \hat{T}$, and $\hat{U}$ are \mycite{Peskin:1991sw} 
\begin{align}
 S=\frac{4 s_W^2 \hat{S}}{\alpha_W},~~ T=\frac{\hat{T}}{\alpha_W},~~U=-\frac{4 s_W^2 \hat{U}}{\alpha_W}
\end{align}


% \more{What is EWPT(in what sense am I using \emph{EWPT}?)}\\
Also many parameters get compared to the observables and compatible results
from LEP experiments:
polarization in $\beta$-decay \cite{Koks:1976zec}, Muon decay\cite{Bardon:1965sd}, 
Pion decay, Neutrion deep inelastic scattering\cite{deGroot:1978feq}, and etc.
% Also, the experiments of $Z$ boson resonance via $e^-e^+$ annihilation has been performed 
% with high accuracy not only to validate the SM but also to suggest guide lines to beyond the SM sector. 
 Especially, $Z$ resononce via $e^+ e^-$ annihlation and $W$ mass measurement 
 pertmit to test sevaeral observables:
 forward/backward asymmetry from the processes $\to \ell\bar{\ell}$ or $\to f\bar{f}$,
 cross section at $Z$ pole, left/right asymmetry, and etc.
 Thess tests altogether usually are regarded as Electroweak Precision Tests(EWPT).

The Standard quantities for these days are \emph{Particle Data Group} \mycite{Agashe:2014kda}. 
\more{21 observables are used for testing Little Higgs models in this thesis 
and they will be shown in the Table somewhere. }


\paragraph{Naturalness ;; maybe this come after little higgs theory part}
To see how much the observables of the model is fine tuned, we need to define a quantity. 
There is no universal definition for the degree of fine-tunning, 
thus, we use the definition of \mycite{ArkaniHamed:2002qy}. % littlest higgs cite

The quadratic divergence of SMlike-Higgs can be cancelled by new particles of similar statistics. 
But the thing is when the new particles mass increases.
Since the loop contributions of the new particles get bigger apart from that of SM particles, 
it is need to be fine tuned for the Higgs mass squared parameter. 
By constraining this, we can avoid fine-tunning as well as give upper bounds how heavy the new particles are. 
For the Littlest Higgs model the cancellon for the Higgs divergence is the top partner, $T^\pm$, and
the dominant log-divergent contribution is caused by $T^\pm$, 
thus it is taken account into:
\begin{align}
 \mu_{exp}^2~=\frac{m_h^2}{2} \nonumber \\
 \delta \mu^2~=~ -\frac{3 \lam_t m_T^2}{8\pi^2}\log\frac{\Lam^2}{m_T^2}
\end{align}
where $\Lam=4\pi f$, $\lam_t$ is the SM top Yukawa coupling. 
We can quantify the fine-tuning as following:
\begin{align}
 \Delta~=~\frac{\mu_{exp}^2}{|\delta \mu^2|}
\end{align}


% \section{the Hierarchy problem}
% 
% % Explanation for introducing Little Higgs and Effective Field Theory
% 
% From the Lagrangian, the scalar particle or fermion can form loops 
% which are quadratically divergent. 
% 
% For example, 
% \begin{align}
%  \cL
% \end{align}
% 

