\chapter{Little Higgs Models}
\label{chap:littlehiggstheory}

\section{Little Higgs mechanism}
Little Higgs models manage the hierarchy problem
that araises from the Higgs couplings 
by the collective symmetry breaking
\mycites{Coleman:1969sm} and \cite{Callan:1969sn}.
Basically Little Higgs models have bigger group structure that the SM group is embedded in. 


Assuming $g_1$ and $g_2$ break each symmetry, $G_1$ and $G_2$, 
the Higgs will be left massless unless $G_1$ and $G_2$ are broekn collectively
because an unbroken symmetry protects Higgs mass term.
And the divergence contribution from the Higgs self-coupling would be logarithmic in the worst case.
The naturalness is guaranteed until $f \sim 4\pi v$. 

\note{this is just what I understood.. but need more details! for Collective Symmetry Beaking}

\section{The Littlest Higgs Model}
\cite{Han:2003wu}
% This model is based on the group structure $SU(5)/SO(5)$. 
\subsection*{Gauge and Scalar sector}
The Littlest Higgs model has $SU(5)/SO(5)$ group structure
which originally is spanned by 24 of generators for $SU(5)$ although it is broken into $SO(5)$ based on the non linear sigma model. 
The vaccum expectation value causing the breaking from $SU(5)$ to $SO(5)$ is
$\Sigma_0$ in the form of $5\times5$ matrix
 \begin{equation}
 \Sigma_0 =  ~ \left( \begin{array}{ccc}
  &  & {\mathbf{1}}_{2 \times 2} \\
  & 1 &  \\
  {\mathbf{1}}_{2 \times 2} &  & 
  \end{array}\right).
  \end{equation}
  And the sigma field that live in $SU(5)$ can be parameterised as
 \begin{equation}
 \Sigma ~=~ e^{i\Pi/f}~\Sigma_0~e^{i\Pi^T/f}~=~ e^{2i\Pi/f}~\Sigma_0. 
 \end{equation}
  with $\Pi=\pi^a X_a$ where $X$ is generators of $SU(5)$.
 Under an $SU(5)$ transformation, it transforms with $V \in SU(5)$ as
  \begin{align}
  \Sigma ~~\to ~~ V \Sigma \tilde{V}^\dagger ~=~V\Sigma V^T .
  \end{align}
 Although the $SU(5)$ symmetry breaking into $SO(5)$ does not change about the vacuum 
 \begin{align}
  U ~\Sigma_0~ U^T = ~\Sigma_0, ~~~~~~~~ U~=~\exp[i\theta^a T_a]. \nonumber
 \end{align}
 We can express $\Sigma_0$ in another form with regard to $U$ and $\theta$ 
 \begin{align}
  \Sigma_0 = \Sigma_0 + i\theta^a (T_a \Sigma_0 + \Sigma_0 T_a^T ) + \cO(\theta^2).
 \end{align}
From the relations above, we obtain conditions for the broken generators $X_a$ and unbroken ones $T_a$ which are
\begin{align}
 \left\{ T_a, \Sigma_0 \right\} =0, \\
 \left[ X_a, \Sigma_0 \right] =0. 
\end{align}


The Littlest Higgs model is categorised as \emph{Product Group models} of Little Higgs Models, which is because
 its gauge symmetry group is $ G_1\times G_2 ~=~ (SU(2)_1\times U(1)_1)\times (SU(2)_2\times U(1)_2)$ embedded in $SU(5)$ as subgroups. 
 These tow copies of $SU(2)\times U(1)$ are supposed to be broken collectively to generate Higgs mass, 
 which results in drawing the limit where the Higgs mass need fine-tunning to $\cO(\unit[10]{TeV})$. \more{need to check if it is true :P}

 %$(SU(2)_1\times U(1)_1) \times (SU(2)_2\times U(1)_2)$ 
  $G_1 \times G_2$ is recovered into the electroweak gague group structure $SU(2) \times U(1)$ at the scale $f$. 
  The generators of $G_1 \times G_2$ are 
%   \begin{equation}\begin{split}       
\begin{align}
  Q_1^a =~ \frac{1}{2}\left( \begin{array}{cc}
  \sigma^a  &  {\mathbf{0}}_{2 \times 3} \\
  {\mathbf{1}}_{3 \times 2} &  {\mathbf{0}}_{3 \times 3} 
  \end{array}\right), ~~~~&
   Q_2^a =~ \frac{1}{2}\left( \begin{array}{cc}
  {\mathbf{0}}_{3 \times 3} & {\mathbf{0}}_{2 \times 3} \\
  {\mathbf{0}}_{3 \times 2} & - \sigma^a
  \end{array}\right) \\
  Y_1 =~ \frac{1}{10}\left( \begin{array}{cc}
  -3 \cdot {\mathbf{1}}_{2\times 2}  &  {\mathbf{0}}_{2 \times 3} \\
  {\mathbf{0}}_{3 \times 2} &  2\cdot {\mathbf{1}}_{3 \times 3} 
  \end{array} \right), ~~~~~~&
  Y2 =~ \frac{1}{10}\left( \begin{array}{cc}
  -2 \cdot {\mathbf{1}}_{3\times 3}  &  {\mathbf{0}}_{2 \times 3} \\
  {\mathbf{0}}_{3 \times 2} &  3\cdot {\mathbf{1}}_{2 \times 2} 
  \end{array} \right)
\end{align}
 where $\sigma^a$ are Pauli matrices.
 
  Since this model based on the non-linear sigma model 
   the kinetic Lagrangian of the model is 
  \begin{equation}
   \cL_{\Sigma}~=~\frac{f^2}{8} {\rm{Tr}}|\cD_\mu \Sigma|^2
  \end{equation}
 where the covariant derivative is given by
  \begin{equation}
   \cD_\mu \Sigma ~=~ \partial_\mu \Sigma - i \sum_j \left[ g_j W_j^a(Q_j^a \Sigma + \Sigma Q_j^{a^T}) + g_j'B_j(Y_j\Sigma + \Sigma Y_j) \right] .
  \end{equation}
  
 From the $SU(5)/SO(5)$ symmetry breaking 14 Nambu-Goldstone bosons appear, which are 
 \begin{align}
   \pi^a X_a =~\left( \begin{array}{ccc}
  \frac{1}{2} X  &  \frac{1}{\sqrt{2}} H^\dagger & \Phi^\dagger \\
  \frac{1}{\sqrt{2}} H & \frac{2}{\sqrt{5}}\eta  & \frac{1}{\sqrt{2}} H^*\\
  \Phi & \frac{1}{\sqrt{2}} H^T &  \frac{1}{2} X
  \end{array}\right)
 \end{align}
where 
\begin{align}
   \pi^a X_a =~\left( \begin{array}{cc} \chi^0 - \frac{1}{\sqrt{5}}\eta &  \sqrt{2}\chi^\dagger \\ \sqrt{2}\chi^- & -\chi^0-\frac{1}{\sqrt{5}}\eta   \end{array}\right), ~~~
   \phi =~\left( \begin{array}{cc} \phi^{++} &  \frac{1}{\sqrt{2}}\phi^+ \\ \frac{1}{\sqrt{2}}\phi^+ \ & \phi^0   \end{array}\right),~~~
   H^T =~\left( \begin{array}{c} \pi^{+} \\ \frac{h+i\pi^0}{\sqrt{2}}\end{array}\right).
\end{align}

 The NGBs are decomposed in $SU(2) \times U(1)$ as $\mathbf{1_0 \oplus 3_0 \oplus 2_{\pm\frac{1}{2}} \oplus 3_{\pm1} }$. 
 Among them, the one corresponding to the SM-like Higgs field is complex doublet $\mathbf{2_{\pm\frac{1}{2}}}$ $H$. 
 And $\Phi$ is the complex triplet $\mathbf{3_{\pm1}}$
 $\eta$ is the real singlet $\mathbf{1_0}$ and $\chi$ refers to the real triplte $\mathbf{3_0}$, those are eaten by the heavy gaue bosons. \more{both? or only $\chi$?}
 
While the symmetry breaking at scale $f$, there is mixing between $G_1$ and $G_2$ and 
under the rotation, the mass eigenstates of the gaue fiels are given by
\begin{align}
 W = sW_1 + cW_2, ~~~~~& W_H=-cW_1 + sW_2 \nonumber \\
 B = s'B_1 + c'B_2, ~~~~~& A_H= -c' B_1+ s'B_2 
\end{align}
with \begin{align}
c=\frac{g_1}{\sqrt{g_1^2 + g_2^2}}, c'=\frac{g_1'}{\sqrt{g_1'^2 + g_2'^2}}
     \end{align}
And the gauge bosons ate NGBs become massive,  
which masses are 
 \begin{align}
  m_{W_H^\pm} = m_{Z_H}=\frac{gf}{2sc}, ~~~~~ m_{A_H} = \frac{g'f}{2\sqrt{5}s'c'}
 \end{align}
where primed ones are of $U(1)$s and uprimed one are of $SU(2)$s.
Also, $g$ and $g'$ are the coefficients of the electroweak gauge group $SU(2)\times U(1)$ and they are 
\begin{align}
 g=g_1 s = g_2c, ~~~~~ g' = g_1's' = g_2' c' 
\end{align}

\more{And $\vev$ correction terms? and Higgs potential?}



% 
% Like other BSM models, 
% the Littlest Higgs model has been suffered from the constraints of the electroweak precision test. 
\subsection*{Fermion sectors}
\more{vectorlike top}\\
\more{yukawa lagrangian of 3rd generation}\\
\more{mass of tops}\\
\more{1loop correction term to higgs mass of CW potential}




\section{The Littlest Higgs Model with T-Parity}
\mycites{Cheng:2003ju} and \cite{Cheng:2004yc},
Just as its name tells us, the Littlest Higgs model with T-parity share the same global and gauge group structure with the Littlest Higgs model 
but \emph{T-parity} is applied additionally. 
The main role of T-parity is to alleviate the electroweak precision constraints on the model.
T-parity is not a quantity related with time symmetry, but it is a discrete symmetry acting on $\cO(\unit[]{TeV})$ scale
behaving similarly to R-parity of Supersymmetric Standard model.  
T-parity is a kind of automorphism between the two groups $SU(2)_1\times U(1)_1 \times SU(2)_2 \times U(1)_2 $.
\more{need more here}



\subsection{T-parity conservation case}
\subsubsection*{Gauge and Scalar sectors}
Under \emph{T-parity}, the generators of $SU(5)$ and $SO(5)$ are transfomed as
\begin{align}
 T^a &\to T^a \nonumber\\
 X^a &\to - X^a.  \nonumber
\end{align}
And the scalar fields can be transfomed as 
\begin{align}
  \Pi ~\to~ -\Omega \Pi \Omega 
  \end{align}
  \begin{align}
  {\rm{where~~}}\Omega =~\left( \begin{array}{ccc} \mathbf{1}_{2\times2}&& \\ &-1& \\ &&\mathbf{1}_{2\times2}  \end{array}\right). \nonumber
\end{align}



Also, under T-parity, the gauge bosons of $G_1$ are converted to ones of $G_2$ and vice versa.
\begin{equation}
 W_{1\mu}^a ~ \xrightarrow{T} ~W_{2\mu}^a ~,~~
 B_{1\mu}~ \xrightarrow{T} ~B_{2\mu}
\end{equation}

This feature of T-parity fixes the mixing angle as $4/\pi$, that is, 
with \begin{align}
s=\frac{g_2}{\sqrt{g_1^2 + g_2^2}}=c=\frac{g_1}{\sqrt{g_1^2 + g_2^2}}=\frac{1}{\sqrt{2}}, \nonumber \\ 
s'=\frac{g_2'}{\sqrt{g_1'^2 + g_2'^2}} = c'=\frac{g_1'}{\sqrt{g_1'^2 + g_2'^2}} = \frac{1}{\sqrt{2}} 
\end{align}
and it fixes also the gauge coupling coefficients as 
\begin{align}
 g_1~=~g_2~=~\sqrt{2}g \nonumber\\
 g_1'~=~g_2'~=~\sqrt{2}g'
\end{align}

\begin{equation}
 m_{W_H}=m_{Z_H}= g f , ~ m_{A_H} = \frac{g'f}{\sqrt{5}}
 \label{eq:LHT:VH}
\end{equation}
where 

\subsubsection*{Fermion sectors}
For applying the collective symmetry breaking to fermion fields,
another state is introduced to the third generation quark doublet 
and this forms an incomplete $SU(5)$ multiplet
\begin{align}
  \Psi = \left( \begin{array}{c}
  i b_L \\
  -it_{1L} \\
  t_{2L} \\
  \mathbf{0}_{2\times1}
  \end{array}\right) 
  = \left( \begin{array}{c}
  q_L \\
  t_{2L} \\
  \mathbf{0}_{2\times1}
  \end{array}\right), 
\end{align}
where $q_L$ is the quark double of the SM. 
Under T-parity, this transforms like
\begin{align}
 \Psi \leftrightarrow - \Sigma_0 \Psi'
\end{align}
where
\begin{align}
  \Psi' = \left( \begin{array}{c}
  \mathbf{0}_{2\times1}\\
  t'_{2L} \\
  i b'_L \\
  -i t'_{1L} 
  \end{array}\right) 
  = \left( \begin{array}{c}
  \mathbf{0}_{2\times1}\\
  t'_{2L} \\
  q'_L 
  \end{array}\right), 
\end{align}



T-parity invariant Lagrangian
\begin{align}
 \mathcal{L}_Y \supset &\frac{\lam_1 f}{2 \sqrt{2}}\eps_{ijk} \eps_{xy} 
 ( \bar{\Psi}_i \Sigma_{jx} \Sigma_{ky} - (\bar{\Psi'})_i \tilde{\Sigma}_{jx} \tilde{\Sigma}_{ky} )t_{1R} \nonumber\\
 & + \lam_2 f (\bar{t}_{2L}t_{2R} + \bar{t'}_{2L} t_{2R} ) + h.c.
\end{align}
With the T-parity eigenstates, 
\begin{align}
 \mathcal{L}_Y \supset &\lam_1 f ( \frac{s_{\Sigma}}{\sqrt{2}}\bar{t}_{1L+}t_{1R+} 
  + \frac{1+c_{\Sigma}}{2}\bar{t}_{2L+}t_{1R+}) \nonumber\\
  & + \lam_2 f ( + \bar{t'}_{2L+} t_{2R+} + \bar{t}_{2L-}t_{2R-} ) + h.c.
\end{align}
where \begin{align}
       q_{L\pm}=\frac{1}{\sqrt{1}}(q_L\mp q'_{L})& \nonumber\\
       t_{2L\pm}=\frac{1}{\sqrt{2}}(t_{2L}\mp t'_{2L}),~&
       t_{2R\pm}=\frac{1}{\sqrt{2}}(t_{2R}\mp t'_{2R}) 
      \end{align}

     
 the mass spectrum for top quarks,
 \begin{align}
  m_{t_{SM}}=m_{t+}=\frac{\lam_1 \lam_2}{\sqrt{\lam_1^2 + \lam_2^2}}v=\frac{\lam_2 R}{\sqrt{1+R^2}}v, \nonumber\\
  m_{T-} = \lam_2 f=\frac{m_t+}{v}\frac{f\sqrt{1+R^2}}{R} ~~, m_{T+}=m_{T-}\sqrt{1+R^2}=\frac{m_{t+}}{v}\frac{f\sqrt{1+R^2}}{R}. 
  \label{eq:LHT:Tpm}
 \end{align}
 
 
Up-type quarks for the first and second generations have the similar Lagrangian
with the top quark except for the vector like quark. 

Yukawa Lagrangian for other fermion fields,
\begin{align}
 \mathcal{L}_{Y} \supset \frac{i\lam_d f}{2\sqrt{2}}\eps_{ij}\eps_{xyz} ( \bar{\Psi'}_x\Sigma_{jy} \Sigma_{jz} X - 
  (\bar{\Psi} \Sigma_0)_x \tilde{\Sigma}_{iy}\tilde{\Sigma_{jz}}\tilde{X}) d_R
\end{align}

$X$ is inserted for gauge invariance, which is a singlet under $SU(2)_{1,2}$ and 
has $U(1)_{1,2}$ charges, $(Y_1, Y_2)=(1/10, -1/10)$. 
$X$ is regarded as $X=(\Sigma_{33})^{-1/4}$ for \emph{Case A} and $X=(\Sigma_{33})^{1/4}$ for \emph{Case B},
where is the $(3,3)$ component of the non-linear sigma model field $\Sigma$. 


To give rise to mass terms for the T-odd fermions without introducing any anomalies, 
another $SO(5)$ multiplet $\Psi_c$ is introduced as
\begin{align}
   \Psi_c = \left( \begin{array}{c}
  i d_c \\
  -iu_{c} \\
  \chi_{c} \\
  i \tilde{d}\\
  -i \tilde{u}_c
  \end{array}\right) 
  = \left( \begin{array}{c}
  q_c \\
  \chi_{c} \\
  \tilde{q}_c
  \end{array}\right), 
\end{align}

$q_c$ is called mirror fermion. 

Its T-parity invariant Lagrangian is 
\begin{align}
 \mathcal{L}_\kappa = 
 -\kappa f (\bar{\Psi'}\xi\Psi_c +\bar{\Psi}\Sigma_0 \Omega \xi^\dagger\Omega\Psi_c ) + h.c. .
\end{align}

This Lagrangian not only adds the T-odd mass terms but also 
imposes new interactions between Higgs boson and up-type partners.
\begin{align}
 \mathcal{L}_\kappa \supset &-\sqrt{2}\kappa f (
 \bar{d}_{L-}\tilde{d}_{c} + \frac{1+c_\xi}{2} \bar{u}_{L-} \tilde{u}_c  \nonumber\\
 &-\frac{s_\xi}{\sqrt{2}}\bar{u}_{L-} \chi_c -\frac{1-c_\xi}{2}\bar{u}_{L-} u_c )
 + h.c. + \cdots
\end{align}
where $c_\xi = \cos((v+h)/\sqrt{2}f)$, $s_\xi=\sin((v+h)/\sqrt{2}f)$. 

We call $\kappa$ as $\kappa_q$ when it is the coupling coefficient 
for the coupling of Higgs and T-odd quarks
and $\kappa_l$ for the coupling of Higgs and T-odd leptons.

The mass spectrum for the T-odd fermions are given at order $\mathcal{O}(v^2/f^2) $ by
\begin{align}
 m_{u-}=\sqrt{2}\kappa_q f (1-\frac{1}{8}\frac{v^2}{f^2}), ~~m_{d-}=\sqrt{2}\kappa_q f
 \label{eq:LHT:qH}
\end{align}

\begin{align}
 m_{\ell_H}=\sqrt{2}\kappa_\ell f
 \label{eq:LHT:lH}
\end{align}

\more{$\ell_H$ mass term! }

\more{1loop correction term to higgs mass of CW potential}


\subsection{T-parity violation case}
\label{sec:LHT:TPV}
\more{More details and explanations here!}
Considering anomalous Wess-Zumino-Witten term as in \mycite{Freitas:2008mq}, 
we can see decays that violate T-parity. 
Thanks to this kind of decay, $A_H$ can decay to the SM particles.
\begin{align} \label{eq:LHT:TPVzz}
  \Gamma(A_H \rightarrow ZZ) = \frac{1}{2\pi}\big( \frac{Ng'}{40\sqrt{3}\pi^2} \big)^2\frac{m_{A_H}^3 m_Z^2}{f^4}
 \big( 1-\frac{4 m_Z^2}{m_{A_H}^2} \big)^{\frac{5}{2}}  \\
 \label{eq:LHT:TPVww} 
 \Gamma(A_H \rightarrow W^+ W^-) = \frac{1}{\pi}\big( \frac{Ng'}{40\sqrt{3}\pi^2} \big)^2 \frac{m_{A_H}^3 m_W^2}{f^4}
 \big( 1-\frac{4 m_W^2}{m_{A_H}^2} \big)^{\frac{5}{2}}  
\end{align}

\begin{align} \label{eq:LHT:TPVff}
 \Gamma&(A_H \rightarrow ff) = 
 \frac{N_{C_f}M_{A_H}}{48\pi}\sqrt{1-\frac{4m_f^2}{M_{A_H}^2}} 
 \big( (c_R - c_L )^2(1-\frac{4m_f^2}{M_{A_H}^2}) + (c_R + c_L)^2(1+\frac{2m_f^2}{M_{A_H}^2}) \big)
\end{align}

\eq{LHT:TPVzz}, \eq{LHT:TPVww}, \eq{LHT:TPVff} are T-parity breaking decay channels 
of $A_H$. 
% 
% \begin{table*}
% \begin{tabular}{c}
% \toprule
% $ \Gamma(A_H \rightarrow ZZ) =$  
% $\frac{1}{2\pi}\big( \frac{Ng'}{40\sqrt{3}\pi^2} \big)^2\frac{m_{A_H}^3 m_Z^2}{f^4} \big( 1-\frac{4 m_Z^2}{m_{A_H}^2} \big)^{\frac{5}{2}}  $ \\
% $ \Gamma(A_H \rightarrow W^+ W^-) =$ 
% $\frac{1}{\pi}\big( \frac{Ng'}{40\sqrt{3}\pi^2} \big)^2 
%  \frac{m_{A_H}^3 m_W^2}{f^4} \big( 1-\frac{4 m_W^2}{m_{A_H}^2} \big)^{\frac{5}{2}} $\\
% $\Gamma(A_H \rightarrow ff) =$
% $ \frac{N_{C_f}M_{A_H}}{48\pi}\sqrt{1-\frac{4m_f^2}{M_{A_H}^2}}
%  \big( (c_R - c_L )^2(1-\frac{4m_f^2}{M_{A_H}^2}) + (c_R + c_L)^2(1+\frac{2m_f^2}{M_{A_H}^2}) \big)$\\
% \bottomrule
% \end{tabular}
% \caption{Coefficients for the }
% \label{tab:benchmarks}
% \end{table*}

where $c_L$ and $c_R$ are the coefficients of the left and right chiral progectors 
and they are shown in \tbl{TPVcoeffi}. 

\begin{table*}%[h]
\begin{center}
 \begin{tabular}{|l|c|c|}
\hline
% Particles & $c_{L,\eps}^f$ & $c_{R,\eps}^f$ \\
Particles & $c_{L}^f$ & $c_{R}^f$ \\
\hline
$A_H e^+ e^-$  & $\frac{9 \hat{N}}{160\pi^2} \frac{v^2}{f^2} g^4 g' (4 + (c_w^{-2}-2t_w^2)^2)$ & $-\frac{9\hat{N}}{40\pi^2}\frac{v^2}{f^2} g'^5$  \\[4pt]
$A_H \bar{\nu}\nu$ & $\frac{9 \hat{N}}{160\pi^2} \frac{v^2}{f^2} g^4 g' (4 + c_w^{-4})$ & 0  \\[4pt]
$A_H \bar{u}_a u_b$  & $-\frac{\hat{N}}{160\pi^2} \frac{v^2}{f^2} g^4 g' (36 + (3c_w^{-2}-4t_w^2)^2)\delta_{ab}$ & $-\frac{\hat{N}}{10\pi^2}\frac{v^2}{f^2} g'^5 \delta_{ab}$  \\[4pt]
$A_H \bar{d}_a d_b$  & $-\frac{\hat{N}}{160\pi^2} \frac{v^2}{f^2} g^4 g' (36 + (3c_w^{-2}-2t_w^2)^2)\delta_{ab}$ & $-\frac{\hat{N}}{40\pi^2}\frac{v^2}{f^2} g'^5 \delta_{ab}$  \\[4pt]
\hline
\end{tabular}
\caption{Coefficients for the $A_H$ TPV decays}
\label{tab:TPVcoeffi}
\end{center}
\end{table*}
% 
\section{The Simplest Little Higgs Model}
\mycite{Schmaltz:2004de}, \mycite{Han:2005ru}

The Simplest Little Higgs model is one of the Little Higgs models where the Little Higgs mechanism is executed in the simply extended group structure,
which is in the form of $SU(N)\times U(1)$. As we can guess from its name, the Simplest Little Higgs model has the minimun value for $N$,
that is, this model has $SU(3)\times U(1)/ SU(2)\times U(1)$ as the group structure.
% The CSB can happen in the simple group structure,
% of which group structure is $SU(N)\times U(1)/ SU(2)\times U(1)$.
% The Simplest Little Higgs model (SLH) is the case where $N$ is the smallest number
% among the CSB happening for the simple groups. 
% % % % % % % % % % % %  영어가 이상. 다시 써봐야징

From the $SU(3)\times U(1)_X / SU(2)_W \times U(1)_Y $ symmetry breaking, 
 5 NGBs appear, which are eaten by $Z',~X^{\mp},~Y^0,$ and $\bar{Y^0}$.
%$Z_H,~W_H^{\pm},~W^{0,0'}$.   Then where does the Higgs come from???????????

To achieve the CSB mechanism, we need doubled of the gauge groups.
\begin{align}
 (SU(3)_1\times U(1)_1 \times SU(3)_2\times U(1)_2)/ (SU(2)\times U(1))^2 \label{eq:gaugegroup}
\end{align}

\subsection*{Gauge and Scalar sectors}
The vacuum expecation values to trigger the symmetry breaking 
for this group structure $(SU(3)\times U(1))^2/ (SU(2)\times U(1))^2$
are $\Phi_1, ~ \Phi_2$, which are not distinguishable in principle, 
but the subscription is for our convinience.
They are given by
\begin{align}
 \Phi_1 = e^{i\Theta f_2/f_1}  ~ \left( \begin{array}{c}
  0 \\
  0  \\
  f_1
  \end{array}\right) ,~~
  \Phi_2 = e^{i\Theta f_1/f_2}  ~ \left( \begin{array}{c}
  0 \\
  0  \\
  f_2
  \end{array}\right) ,~~ 
\end{align}
where 
\begin{align}
 \Theta = \frac{1}{f} ~ \left[ \left( \begin{array}{ccc}
  0 & 0 & h\\
  0 & 0 &  \\
  h^\dagger & & 0
  \end{array}\right) 
  + \frac{\eta}{\sqrt{2}} \left( \begin{array}{ccc}
  1 & 0 & 0 \\
  0 & 1 & 0 \\
  0 & 0 & 1
  \end{array}\right)   \right], ~~~~ 
  h = \left( \begin{array}{c}
              h^0\\
              h^-
             \end{array}    \right) . 
\end{align}

$h^T=(h^0, h^-)$ is complex scalar doublet corresponding to the Higgs.
$f_1, f_2$ are the symmetry breaking scale and 
they can be redefine as $f=\sqrt{f_1^2 + f_2^2}$ and $\tan \beta~=~f_1 /f_2$.
Also, we frequently use $\cos\beta=f_1/\sqrt{f_1^2 + f_2^2}$ ($c_\beta$) 
and $\sin\beta=f_2/\sqrt{f_1^2 + f_2^2}$ ($s_\beta$).



\paragraph{Scalar particles}
\more{SLH paper : \mycite{Schmaltz:2004de} and pseudoaxion: \mycite{Kilian:2004pp}}\\
The Higgs potential can be written like
\begin{align}
 V=m^2 h^\dagger h + \lam (h^\dagger h )^2,
\end{align}
where $m$ is required to be negative for Electroweak symmetry breaking. 

We extend this potential for this model with the scalar fields $\Phi_1, \Phi_2$
\begin{align}
 V~&=~\mu^2 \Phi_1^\dagger \Phi_2 + {\rm{h.c.}}\nonumber\\
  & = 2s_\beta c_\beta \mu^2\cos\left( \frac{\eta}{\sqrt{2}s_\beta c_\beta} \right)
  \left[ 1-\frac{1}{2s_\beta^2 c_\beta^2}(h^\dagger h) + \frac{f(h^\dagger h)^2}{24s_\beta^3 c_\beta^3}+... \right].
\end{align}


In the SLH model, the scalar fields are given the radiative correction to the mass terms
by Coleman-Weinberg mechanism that generates one-loop correction to the Higgs mass $\delta m^2$
as well as to quartic couplings. 
\begin{align}
 \delta m^2 = -\frac{3}{8\pi^2}\left[ \lam_t^2 m_T^2 \log\left(\frac{\Lam^2}{m_T^2}\right)
  - \frac{g^2}{4} m_{X}^2\log\left(\frac{\Lam^2}{m_X^2}\right)
  - \frac{g^2}{8}(1+\tan^2\theta_w)m_{Z'}^2\log\left(\frac{\Lam^2}{m_{Z'}^2}\right) \right]
\end{align}


Thus, in total the potential for the scalar fields can be written as 
\begin{align}
 V &= \left(\delta m^2 + \frac{\mu^2}{s_\beta c_\beta}\right) h^\dagger h + \frac{\mu^2}{s_\beta c_\beta}\frac{\eta^2}{2}
  -\left(\frac{\mu^2}{12(s_\beta c_\beta f)^2} - \delta \lam\right)(h^\dagger h)^2 \nonumber\\
  &~~~~~- \frac{\mu^2}{12(s_\beta c_\beta f)^2} \left(\frac{\eta^4}{4} + \frac{3(h^\dagger h)\eta}{2} \right)+ ...
\end{align}
where $\mu$ is free parameter that is introduced to manage the Higgs mass 
and $s_\beta,~c_\beta$ are the quantities about the ratio of the symmetry breaking scale for the extended group.
Clearly, there are the Higgs and the $\eta$ mass terms,
\begin{align}
 m_\eta^2 =& \frac{\mu^2}{s_\beta c_\beta} \label{eq:etamass} \\
 m_H^2 =& -2(\delta m^2 + m_\eta^2) \label{eq:higgsmass}
\end{align}





\paragraph{The heavy gauge boson mass terms} are given 
from the expansion of the nonlinear sigma model Lagrangian, which are in the forms of 
% Using $f$ and the weak mixing angle $\theta_w$, heavy gauge boson mass terms are expressed by
\begin{align}
 M_{X^\mp} = \frac{g f}{\sqrt{2}}, ~~~ M_{Z_H} = g f \sqrt{\frac{2}{2-\tan\theta_w}}.
\end{align}
where $\theta_W$ is the weak mixing angle and the covariant derivative $D_\mu$ is given by
\begin{align}
 \cD_\mu~=~(\partial_\mu + i g A^a T_\mu^a + i g_x B_\mu)  .
\end{align}
where $B_\mu$ is the gauge boson of $U(1)$ and $A_a T^a$ includes $SU(3)$ gauge bosons as its components,
\begin{align}
 A^a T^a = \frac{A^3}{2}\left( \begin{array}{ccc} 1 &  & \\ & -1 &  \\  & & 0 \end{array}\right) 
 + \frac{A^8}{2\sqrt{3}}\left( \begin{array}{ccc} 1 &  & \\ & 1 &  \\  & & -2 \end{array}\right) 
 + \frac{1}{\sqrt{2}}\left( \begin{array}{ccc} & W^+ &Y^0\\W^- & &X^-\\ \bar{Y}^0 &X^+& \end{array}\right) 
\end{align}


In total, this model have 2 symmetry breaking procedure; one is into the SM gauge group structure $SU(2)\times U(1)$ 
and another is the Electroweak symmetry breaking.
We can find the trajactories of these symmetry breaking from their expressions in terms of the mass eigenstates. 

Before Electroweak symmetry breaking, 
The gauge bosons $Y$

This model is based on the nonlinear sigma model, accordingly 
the Lagrangian for scalar and gauge boson part is given by 
\begin{align}
 \cL = \big| \big(\partial_\mu+igA_\mu^a T^a -\frac{ig_x}{3} B_\mu^x \big) \Phi_i \big|^2
\end{align}


The heavy gauge bosons' expressions are 
\begin{align}
 Z_0'=\frac{\sqrt{3}g A^8 + g_x B^x}{\sqrt{3 g^2 + g_x ^2}} 
 = \frac{1}{\sqrt{3}}(\sqrt{3-\tan^2\theta_w}A^8 + \tan\theta_w B^x) \\
 B=\frac{-g_x A^8 + \sqrt{3}g B^x}{\sqrt{3 g^2 + g_x ^2}} 
 = \frac{1}{\sqrt{3}}(-\tan\theta_w A^8 + \sqrt{3-\tan^2\theta_w} B^x) \\
\end{align}

g is the coupling constant of $SU(2)$ that is equal to that of $SU(3)$.
$g_x$ refers to the coupling constant of $U_X(1)$ which is mixed as 
$g_x = \frac{g \tan\theta_w}{\sqrt{1-\tan^2\theta_w / 3}}$

Hypercharges follow $Y=-\frac{1}{\sqrt{3}}T^8 + Q_X$
with the eigenvalue of 8th generator of $SU(3)$,  $T^8=\frac{1}{2\sqrt{3}}{\rm{diag}}(1,1,-2)$,
where $Y$ is the hypercharge of $U_Y(1)$ and $Q_X$ is that of $U_X(1)$.



\subsection*{Fermion sectors}
Thanks to its peculiar group structure, we have to find the proper way to 
decompose in for fermions. 
There are two way for this: \emph{universal} and \emph{anomaly-free}.
It is better to focus on anomaly-free way because this way makes the anomalies cancelled, 
as well as univeresal way is already excluded.
\more{Why it is called \emph{anomaly-free} }

The quarks of the third generation and the leptons are expressed with the triplets of $SU(3)$ and also in the singlets. 
\begin{align}
 L_m^T=(\nu,e,iN)_m,~~&~~~ ie_m^c, iN_m~~&~~{m\in\{1,2,3\}} \nonumber\\
 Q_3^T= (t,b,iT), ~~&~~~it^c, ib^c,iT^c ~~&~ 
\end{align}
where $m$ is the generation index and $N_m$ is new heavy particle as the partner of neutrino. 
The heavy vector-like top and bottom quark have electric charge $+2/3$.
These are same way to decompose with that of \emph{universal}.
With regard to anomaly-free decomposition, the first and second generation quarks are given by 
\begin{align}
 Q_1^T= (d,-u,iD), ~~ &~~  id^c, iu^c, iD^c ~~& ({\rm{anomaly~free}})\nonumber\\
 Q_2^T= (s,-c,iS),~~ &~~ is^c, ic^c, iS^c ~~ & ~~
\end{align}
which have electirc charge $-1/3$.
\more{need quantum number charge table for the all quarks }

\subsubsection*{Lepton mass}
The mass terms for leptons are generated from the Lagrangian
\begin{align}
 \cL ~=~ i \lam_{N_m} N_m^c \Phi_2^\dagger L_m 
 + \frac{i \lam_e^{mn}}{\Lam}e_m^c \eps_{ijk}\Phi_1^i \Phi_2^j L_n^k + {\rm{h.c.}}, 
\end{align}
where $n,~m$ are the indices for generations of leptons. 
The heavy neutrino partners $N_m$ have mass term
\begin{align}
 M_{N_m} ~=~ \lam_{N_m}s_\beta f \nonumber
\end{align}
from the first term of the Lagrangian. 
The second term has 5 spacetime dimension, which is normalised by $\Lam$ cutoff scale. 
Also, this operator give rise to mass terms to charged leptons 
by the mixing matrix $V_{im}^\ell$ between $e_i$ and $N_m$ similarly with the CKM mixing,
which can happen in association with the gauge boson $X^-$ in the $X^- \bar{e_i} N_m$ coupling 
that can be found in the Lagrangian as
\begin{align}
 \cL \supset -\frac{g}{\sqrt{2}} V_{im}^\ell X_\mu^- \bar{e_i}\gamma^\mu P_L N_m~~ .
\end{align}
But, the mixing is suppressed by GIM mechanism and as diagonalising $V_{im}^\ell$, it vanishes.

\more{lepton flavor violation}


After the EWSB, mixing between neutrinos $\nu_{m0}$ and the heavy neutrino partners $N_{m0}$ appears at order $v/f$.
\more{more about the mixing and mixed basis in detail}
\begin{align}
 N_{m0} = N_{m} + \delta_\nu V_{mi}^{\ell\dagger} \nu_i, ~~~~~~~~
 \nu_{i0} = (1-\frac{1}{2}\delta_\nu^2) \nu_i - \delta_\nu V_{im}^\ell N_m .
\end{align}



\more{interations between heavy gauge bosons $X\pm, Y^0$ and $\bar{Y^0}$}


\subsubsection*{Lepton couplings to gauge bosons} 
can be found from the fermion kinetic terms of the Lagrangian. 
\begin{align}
 \cL~=~\bar{\psi}i\cD_mu \gamma^mu \psi, ~~~~~~~~
 \cD = \partial + ig A^a T^a + ig_x Q_x B^x, 
\end{align}



\subsubsection*{Quark mass}

As \emph{universal} embedding is excluded already, we are discussing only about \emph{anomaly-free} embedding.
Because the third generation quarks sector have different expression from the other two generations,
the Lagrangian of the Yukawa interacion for quarks have also different expression according to the generations,
which is given by
\begin{align}
 \cL_3 ~=&~~ \lam_1^t it_1^c \Phi_1^\dagger Q_3 + \lam_2^t it_2^c \Phi_2^\dagger Q_3 
 + \frac{\lam_b^m}{\Lam} id_m^c \eps_{ijk}\Phi_1^i \Phi_2^j Q_3^k + {\rm{h.c.}} \nonumber \\
 \cL_{1,2}~ =&~~ \lam_1^{dn} id_i^{nc} Q_n^T \Phi_1 + \lam_2^{dn} id_2^{nc} Q_n^T \Phi_2 
 + \frac{\lam_u^{mn}}{\Lam} iu_m^c \eps_{ijk} \Phi_1^{*i} \Phi_2^{*j} Q_n^k + {\rm{h.c.}}
 \end{align}
where $n= 1,2$ that is index of first and second generation; $m=1,2,3$ referring also to the generation indeices; and $i,j,k=1,2,3$. %\eq{gaugegroup} 
% $n$ refers to the indices of the gauge group $(SU_1(3)\times U_1(1) \times SU_2(3)\times U_2(1))$.
The linear combinations of $t_{1,2}^c$ form the states for heavy top partner $T^c$ and $t^c$ and similarly 
$d_{1,2}^{nc}$ form that of $D^c,~d^c$ with $n=1$ and $S^c,~s^c$ with $n=2$.

The heavy quark $T^c$ state are orthogonal to $t^c$ states as 
\begin{align}
 T^c~=~\frac{\lam_1^t c_\beta u_1^c + \lam_2^t s_\beta u_2^c}{\sqrt{\lam_1^{t2} c_\beta^2 + \lam_2^{t2} s_\beta^2}}, ~~~~~ 
 t^c~=~\frac{-\lam_2^t s_\beta u_1^c + \lam_1^t c_\beta u_2^c}{\sqrt{\lam_1^{t2} c_\beta^2 + \lam_2^{t2} s_\beta^2}},
\end{align}
where the $\beta$ of $s_\beta,~c_\beta$ is $\arctan(f1/f2)$ and $\lam_{1,2}^t$ is the Yukawa coupling constant of top of the group $SU_{1,2}(3)$.
The coupling of $T^c$ to $T$ gives rise to its mass leaving $t^c$ massless
\begin{align}
 M_T = ~~f \sqrt{\lam_1^{t2} c_\beta^2 + \lam_2^{t2} s_\beta^2}.
\end{align}

Also, $D^c,~d^c$ consist of $d_{1,2}^{1c}$($d_{1,2}^c$) and $S^c,~s^c$ consist of $d_{1,2}^{2c}$($s_{1,2}^{c}$) which are expressed by
\begin{align}
 D^c~=~\frac{\lam_1^d c_\beta d_1^{c} + \lam_2^d s_\beta d_2^{c}}{\sqrt{\lam_1^{d2} c_\beta^2 + \lam_2^{d2} s_\beta^2}}, ~~~~~
 d^c~=~\frac{-\lam_1^d s_\beta d_1^{c} + \lam_1^d c_\beta d_2^{c}}{\sqrt{\lam_1^{d2} c_\beta^2 + \lam_2^{d2} s_\beta^2}} \nonumber\\
% \end{align}
% \begin{align}
 S^c~=~\frac{\lam_1^s c_\beta s_1^{c} + \lam_2^s s_\beta s_2^{c}}{\sqrt{\lam_1^{s2} c_\beta^2 + \lam_2^{s2} s_\beta^2}}, ~~~~~
 s^c~=~\frac{-\lam_1^s s_\beta s_1^{c} + \lam_1^s c_\beta s_2^{c}}{\sqrt{\lam_1^{s2} c_\beta^2 + \lam_2^{s2} s_\beta^2}} .
%  S^c~=~\frac{\lam_1^d c_\beta d_1^{2c} + \lam_2^d s_\beta d_2^{2c}}{\sqrt{\lam_1^{d2} c_\beta^2 + \lam_2^{d2} s_\beta^2}}, ~~~~~
%  s^c~=~\frac{-\lam_1^d s_\beta d_1^{2c} + \lam_1^d c_\beta d_2^{2c}}{\sqrt{\lam_1^{d2} c_\beta^2 + \lam_2^{d2} s_\beta^2}} 
\end{align}
And each $D^c$ and $S^c$ couples to $D$ and $S$, which give them masses as 
\begin{align}
 M_D=f\sqrt{\lam_1^{d2} c_\beta^2 + \lam_2^{d2} s_\beta^2},~~~M_S=f\sqrt{\lam_1^{s2} c_\beta^2 + \lam_2^{s2} s_\beta^2}
\end{align}
again leaving $d^c$ and $s^c$ as massless states.

After EWSB, the Lagrangian about the quark mass terms are 
\begin{align}
  \cL_{\rm{up~mass}}~=~& -M_T T^c T + \frac{v}{\sqrt{2}} \frac{s_\beta c_\beta (\lam_1^{t2}-\lam_2^{t2})}{\sqrt{\lam_1^{t2} c_\beta^2 + \lam_2^{t2}s_\beta^2}}T^c t
         - \frac{v}{\sqrt{2}} \frac{\lam_1^t \lam_2^t}{\sqrt{\lam_1^{t2} c_\beta^2 + \lam_2^{t2} s_\beta^2}}t^c t \nonumber\\
         &+\frac{v}{\sqrt{2}}\frac{f}{\Lam}\lam_u^{mn}u_m^c u_n + {\rm{h.c.}} \label{eq:massLagup}\\
 \cL_{\rm{down~mass}}~=~& -M_D D^c D - \frac{v}{\sqrt{2}} \frac{s_\beta c_\beta (\lam_1^{d2}-\lam_2^{d2})}{\sqrt{\lam_1^{d2} c_\beta^2 + \lam_2^{d2}s_\beta^2}}D^c d
         + \frac{v}{\sqrt{2}} \frac{\lam_1^d \lam_2^d}{\sqrt{\lam_1^{d2} c_\beta^2 + \lam_2^{d2} s_\beta^2}} d^c d \nonumber\\
                        & -M_S S^c S - \frac{v}{\sqrt{2}} \frac{s_\beta c_\beta (\lam_1^{s2}-\lam_2^{s2})}{\sqrt{\lam_1^{s2} c_\beta^2 + \lam_2^{s2}s_\beta^2}}S^c s
         + \frac{v}{\sqrt{2}} \frac{\lam_1^s \lam_2^s}{\sqrt{\lam_1^{s2} c_\beta^2 + \lam_2^{s2} s_\beta^2}} s^c s \nonumber\\
         & +\frac{v}{\sqrt{2}}\frac{f}{\Lam}\lam_b^{m} d_m^c b + {\rm{h.c.}} \label{eq:massLagdown}
\end{align}
 where $u_n$=u,c; $u_m^c=u^c,c^c,t^c,T^c$; and $d_m^c=d^c,s^c,b^c,D^c,S^c$.
 % Now let us have a look details of the last term of \eqs{massLagup}{massLagdown}. 
The terms with coefficients $\lam_u^{mn}$ and $\lam_b^m$ cause the mixing of terms between different generation quarks in up and down sectors leading to the CKM matrix. 
% And similarly, the SM quarks mass eigenstates are mixed with that of heavy quarks $D,~S$ and $T$ leading to the analogous matrix, 
% two of which play roles in the interactions between quarks and gauge bosons.
% Additionally, mixing terms between heavy left-handed quarks $D,S,T$ and SM quarks are caused by EWSB.
To convert the flavour basis of quarks to the mass basis ones, we need the CKM matrix;
\begin{align}
  V^u \left( \begin{array}{c}  u'\\  c' \\  t' \end{array}\right) ~=~\left( \begin{array}{c}  u \\  c \\  t \end{array}\right), ~~~~~~~~~
  V^d \left( \begin{array}{c}  d'\\  s' \\  b' \end{array}\right) ~=~\left( \begin{array}{c}  d \\  s \\  b \end{array}\right) 
\end{align}
where primed fields, $u'$s, are referring to flavour basis and unprimed ones, $u$s, are mass bases. $V^u$ and $V^d$ form the CKM matrix:
\begin{align}
 V^{\rm{CKM}}~=~V^u V^{d\dagger}
\end{align}
% \more{from\cite{Han:2005ru}, there are two physically meaningful mixing matrices.... MAYBE dl ekdmadp skdhsms $\delta$fkd $V_{ij}$ dlfjsrjsrk?}
To prevent from confused about mixing matrices for quarks, it would be better to mention briefly about them.
There are two mixing matrices; one is the CKM matrix $V^{CKM}$ we just discussed, which includes mixing informaiton among the flavour and mass bases.
And the other one is caused from Electroweak symmetry breaking (EWSB) among the interaction basis and mass eigenstate. 

% {The mixing terms} between the heavy left-handed quarks $D,S,T$ and the SM qaurks are induced by EWSB. 
\paragraph{The mixing of up quark sector} is induced by EWSB as just mentioned.
By the mixing terms of $T$ and $u,~c,~t$ in \eq{massLagup}, $SU(3)$ symmetry is violated. 
Let us have a look in detail about the mass bases of quarks.
The $SU(3)$ state $T_0$ before EWSB can be expressed in terms of the mass eigenstate $T$ with mixing terms induced by EWSB 
as $T_0~=~T+\delta_{u_i}u_i'$ with $i=1,2,3$, of which $\delta_{u_i}$ are given by
\begin{align}
 \delta_u = \frac{v}{\sqrt{2}\Lam}\frac{\lam_u^{T^c u}}{\sqrt{\lam_1^{t2}c_\beta^2 + \lam_2^{t2} s_\beta^2}}, ~~
 \delta_c = \frac{v}{\sqrt{2}\Lam}\frac{\lam_u^{T^c c}}{\sqrt{\lam_1^{t2}c_\beta^2 + \lam_2^{t2}s_\beta^2}},~~
 \delta_t = \frac{v}{\sqrt{2}f}\frac{s_\beta c_\beta (\lam_1^{t2} - \lam_2^{t2} )}{\sqrt{\lam_1^{t2}c_\beta^2 + \lam_2^{t2}s_\beta^2}}.
\end{align}

To have the mixing terms of $T$ to $u$ and $c$ suppressed, we take the couplings $\lam_u^{T^c u}$ and $\lam_u^{T^c c}$ small enough to neglect these mixing terms.
With neglecting them, the relation of the mass eigenstate and the interaction ones becomes
\begin{align}
T_0 = T+ \Delta_{u_i} u_i, ~~~~~~~~~~~~~~~~ \Delta_{u_i} = V_{ij}^{u*} \delta_{u_j} \simeq V_{i3}^{u*} \delta_t. 
\end{align}
The up quarks can be expressed as 
\begin{align}
 u_{i0} = (1-\frac{1}{2}|\Delta_{u_i}|^2)u_i - \Delta_{u_i} T,
\end{align}
where we expect corrections to $\Delta_{u_i}$ from the precisely measured couplings of quarks to the $W$ boson.


\paragraph{In the down quark sector}, the story is similar. 
The $SU(3)$ state $D_0$ and $S_0$ which are the state before EWSB can be written in terms of the mass eigenstates $D$, $S$ and the SM fermions
in the interacion basis as 
\begin{align}
 D_0=D+\delta_{Dd_i}d_i', ~~~~~~S_0=S+\delta_{Sd_i} d_i'
\end{align}
where $i=1,~2,~3$ and
\begin{align}
 &\delta_{Dd} =\frac{-v}{\sqrt{2}f}\frac{s_\beta c_\beta (\lam_1^{d2} - \lam_2^{d2} )}{\sqrt{\lam_1^{d2}c_\beta^2 + \lam_2^{d2}s_\beta^2}},~~~~~~~~
 \delta_{Ds}=0,
 &\delta_{Db} =\frac{v}{\sqrt{2}\Lam}\frac{\lam_b^{D^c}}{\sqrt{\lam_1^{d2}c_\beta^2 + \lam_2^{d2}s_\beta^2}},\\ 
 &\delta_{Sd}=0,  ~~~~~~~~
 \delta_{Ss} =\frac{-v}{\sqrt{2}f}\frac{s_\beta c_\beta (\lam_1^{s2} - \lam_2^{s2} )}{\sqrt{\lam_1^{s2}c_\beta^2 + \lam_2^{s2}s_\beta^2}} 
 &\delta_{Sb} =\frac{v}{\sqrt{2}\Lam}\frac{\lam_b^{S^c}}{\sqrt{\lam_1^{s2}c_\beta^2 + \lam_2^{s2}s_\beta^2}}.  
\end{align}
Thanks to the collective symmetry breaking mechanism, the mixing between $D-s$ and $S-d$ became zero. 
\more{a consequence of the collective breaking mass generation for d and s in the D,S mass basis fkd wjrjfkd rkxdmsakfdlswlwha todrkrgoqhkidgk gkfemt}
To let the mixing of $D$, $S$ with $b$ quark suppressed, we assume $\lam_b^{D^c}$ and $\lam_b^{S^c}$ to be small. 
Also, to be consistent with the SM, one of $\lam_{1,2}^d$ and $\lam_{1,2}^s$ are expected to be small from \eq{massLagdown},
which can result in the tiny mixing effects in the down quark sector like the mixing effects in the neutrino sector.
Then, \begin{align}
       \delta_{Dd} \simeq \delta_{Ss} \simeq \frac{v}{\sqrt{2}t_\beta f} = -\delta_\nu .
      \end{align}
      

 The $D$      and $S$ states become 
 \begin{align}
  D_0 = D + \Delta_{Dd_i}d_i , ~~~~~ \Delta_{Dd_i}=V_{ij}^{d*} \delta_{Dd_j} \simeq -V_{i1}^{d*} \delta_\nu \nonumber \\
  S_0 = S + \Delta_{Sd_i}d_i , ~~~~~ \Delta_{Sd_i}=V_{ij}^{d*} \delta_{Sd_j} \simeq -V_{i2}^{d*} \delta_\nu 
 \end{align}
 Considering the mixings, the down quarks in the mass basis become 
 \begin{align}
  d_{i0}=(1-\frac{1}{2}|\Delta_{Dd_i}|^2 - \frac{1}{2}|\Delta_{Sd_i}|^2)d_i -\Delta_{Dd_i}D - \Delta_{Sd_i}S, 
 \end{align}
 where $|\Delta_{Dd_i}|^2$ and $|\Delta_{Sd_i}|^2$ terms are left of order $v^2/f^2$. 

      

\subsubsection*{Couplings of quarks to scalar particles }

% Now we have a look at the coulings between new heavy quark pairs $D,~S,~T$ and scalar particles $H~\eta$; o

\paragraph{The couplings of $T,~D,$ and $S$ quark pairs to the scalars, $H$ and $\eta$} cause the flavour changing couplings \more{ explain moremore}

\begin{align}
 \cL_{T^c T} \simeq &(H T^c T) \frac{v}{f} [(\lam_1^{t2}s_\beta^2 + \lam_2^{t2}c_\beta^2)\frac{f}{2M_T} - s_\beta^2 c_\beta^2(\lam_1^{t2} - \lam_2^{s2})^2\frac{f^3}{2M_T^3}  ]\nonumber\\
  & (i\eta T^c T) s_\beta c_\beta (\lam_1^{t2} - \lam_2^{s2}) \frac{f}{2M_T} + {\rm{h.c.}} 
\end{align}
with neglecting $\lam_u^{T^c u}$ and $\lam_u^{T^c c}$.

Similarly, $D$ and $S$ quark pairs 
\begin{align}
 \cL_{D_m^c D} \simeq (i\eta D^c D) \frac{c_\beta \lam_2^{d}}{\sqrt{2}}  + (i\eta S^c S) \frac{c_\beta \lam_2^s}{\sqrt{2}}+ {\rm{h.c.}},  
\end{align}
where \more{why we ignore the terms with $\lam_1^d$}


\paragraph{The couplings of one $T$ quark and one SM quark} are 
\begin{align}
 \cL \simeq (HT^c u_i) [s_\beta c_\beta (\lam_1^{t2} - \lam_2^{t2})\frac{f}{\sqrt{2}M_T} V_{i3}^{u*} ] 
 - (i\eta t^c T) [\frac{\lam_1^t \lam_2^t}{\sqrt{2}M_T} ] + {\rm{h.c.}}
\end{align}
where 

\subsubsection*{Couplings of quarks to gauge bosons } 
% \more{If I assume ther is no flavor mixing between different generations of fermions, then the mixing becomes much simpler like in \emph{Andreathesis}.}
%From \mycite{Han:2005ru}, 
Assuming that there is no flavour mixing between fermions, we can check the possible $Z'$ couplings to fermions in \tbl{Zff}.
Interestingly, $Z'$ boson induced from $SU(3)\times U(1)$ like the Simplest Little Higgs has the interacion couplings as free parameters. \more{need to check if this is correct.} 
If the group is bigger than $SU(3)\times U(1)$, for example $SU(4)\times U(1)$, the coupling coefficients of $Z'$ and $Z''$ become constraints each other.
Besides, including the flavour changing sectors, the coupling terms of $Z'$ to quarks are
\begin{align}
 \cL_{Z'}~\supset~ -\frac{g}{c_W}\frac{Z_\mu'}{\sqrt{3-4s_W^2}}\huge[ (-&\frac{1}{2}+\frac{2}{3}s_W^2)(\bar{u_i}\gamma^\mu u_i  + \bar{d_i}\gamma^\mu d_i) \nonumber \\
 &+ (1-s_W^2)(V_{i3}^u V_{3j}^{u\dagger}\bar{u_i}\gamma^\mu u_j + V_{i3}^d V_{3j}^{d\dagger}\bar{d_i}\gamma^\mu d_j) \huge] 
\end{align}
where the last two terms are flavour mixing terms for up- and down-type quarks with $V_{i3}^u V_{3j}^{u\dagger}$ and $V_{i3}^d V_{3j}^{d\dagger}$.

\begin{table} %[h]
\begin{center}
  \begin{tabular}{| c | c |}
    \hline
    $Z'\bar{t}t$ &$-\frac{ig}{c_W\sqrt{3-4s_W^2}}[(\frac{1}{2}-\frac{1}{3}s_W^2) P_L + \frac{2}{3}s_W^2P_R]$ \\[5pt] \hline
    $Z'\bar{b}b$ &$-\frac{ig}{c_W\sqrt{3-4s_W^2}}[(\frac{1}{2}-\frac{1}{3}s_W^2) P_L - \frac{1}{3}s_W^2P_R]$  \\[5pt] \hline
    $Z'\bar{u}u$ &$-\frac{ig}{c_W\sqrt{3-4s_W^2}}[(-\frac{1}{2}+\frac{2}{3}s_W^2) P_L + \frac{2}{3}s_W^2P_R]$  \\[5pt] \hline
    $Z'\bar{d}d$ &$-\frac{ig}{c_W\sqrt{3-4s_W^2}}[(-\frac{1}{2}+\frac{2}{3}s_W^2) P_L - \frac{1}{3}s_W^2P_R]$  \\[5pt] \hline
    $Z'\bar{e}e$ &$-\frac{ig}{c_W\sqrt{3-4s_W^2}}[(-\frac{1}{2}-s_W^2) P_L - s_W^2P_R]$  \\[5pt] \hline    
    $Z'\bar{\nu}\nu$ &$-\frac{ig}{c_W\sqrt{3-4s_W^2}}[(-\frac{1}{2}-s_W^2) P_L ]$  \\[5pt] \hline    
  \end{tabular}
  \caption{$Z'$-fermions interactions ignoring the flavour misalignments}
  \label{tab:Zff}
\end{center}
 \end{table}
        


The couplings of $X^\mp,~Y^0$ and $\bar{Y^0}$ to fermions are shown in \tbl{XYff} when flavour misalignments and CKM mixing are ignored.
On the other hand, when taking into account the flavour misalignment and the CKM mixing, 
the Lagrangian is given by
\begin{align}
 \cL_{X,Y}~=~ -\frac{g}{\sqrt{2}}\huge{\{} &iX_\mu^- \bar{d}_i\gamma^\mu [V_i3^d T + (\Delta_{uj}V_{i3}^d + \Delta_{Dd_i}^* V_{j1}^{u*} + \Delta_{Sd_i}^* V_{j2}^{u*})u_j] \nonumber \\ 
 & +iX_\mu^+ \bar{u}_i\gamma^\mu V_{ij}^u D_j + i Y_\mu^{0} \bar{u}_i\gamma^\mu (V_{i3^u} T + \Delta_{uk} V_{i3}^u u_k) \nonumber \\ 
 & +i \bar{Y}_\mu^0\bar{d}_i\gamma^\mu [V_{ij}^d D_j + (\Delta_{Dd_k}V_{i1}^d +\Delta_{Sd_k}V_{i2}^d)d_k  ]~ +~{\rm{h.c.}} \huge{\}}
\end{align}
\more{Should explain what $\Delta$s are. }
        
        
\begin{table} %[h]
\begin{center}
  \begin{tabular}{| c | c | }
    \hline
    $X_\mu^-\bar{b}t$ &$\frac{g}{\sqrt{2}} \delta_t \gamma_\mu P_L $ \\ \hline
    $X_\mu^-\bar{d}u$ &$\frac{g}{\sqrt{2}} \delta_\nu \gamma_\mu P_L $ \\ \hline
    $X_\mu^-\bar{e}\nu$ &$\frac{g}{\sqrt{2}} \delta_\nu \gamma_\mu P_L $ \\ \hline    
        $Y_\mu^0\bar{t}t$ &$\frac{g}{\sqrt{2}} \delta_t \gamma_\mu P_L $ \\ \hline
    $Y_\mu^0\bar{u}u$ &$0$ \\ \hline
    $Y_\mu^0\bar{d}d$ &$\frac{g}{\sqrt{2}} \delta_\nu \gamma_\mu P_L $ \\ \hline
    $Y_\mu^0\bar{e}e$ &$0$ \\ \hline
    $Y_\mu^0\bar{\nu\nu}t$ &$\frac{g}{\sqrt{2}} \delta_\nu \gamma_\mu P_L $ \\ \hline
    $Y_\mu^0H\eta$ &$\frac{ig}{2\sqrt{2}} (p_\eta - p_H)_\mu $ \\ \hline
  \end{tabular}
  \caption{$X,~Y$-fermions interactions neglecting flavour misalignments}
  \label{tab:XYff}
\end{center}
 \end{table}
        


\subsection{Phenomenology of SLH}
\subsubsection*{Pseudo-axion}
\cite{Kilian:2004pp} \more{is the reference for this section. introduce some content of this paper
and then I can put some plot of this pseudo scalar}
If one of $U(1)$ is not gauged, another pseudo-Goldston boson is left, not eaten.
This is $\eta$. 
$\eta$ earns its mass from Coleman-Weinberg potential which is related term 
with SM-like Higgs mass. 
\begin{align}
 M_\eta = Coupling\times \log (\frac{M_t^2}{M_{Z_H}^2}) ...
\end{align}

Since we know Higgs mass, another constraint for this is given. 



\subsubsection*{Heavy gauge bosons}

$Y^0,~\bar{Y^0}$ and $X^\mp$($Z_H$ and $W_H$) appear in other bsm scenarios frequently. 

