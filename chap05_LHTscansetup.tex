% %%%%%%%%%%%%%%%%%%%%%%%%%%%%%%%%%%%%%%%%%%%%%%%%%%%%%%%  
\chapter{Phenomenology of LHT at the LHC}
\label{chap:LHTsetup}
% %%%%%%%%%%%%%%%%%%%%%%%%%%%%%%%%%%%%%%%%%%%%%%%%%%%%%%%
% 
% 
\section{ processes, reasons, production cross section, branching ratio}

\paragraph{processes with new particles}
We consider 6 processes for studying this model.  
\begin{enumerate}
\item $p p \rightarrow q_H q_H, q_H \bar{q}_H, \bar{q}_H \bar{q}_H$,
\item $p p \rightarrow q_H V_H$,
\item $p p \rightarrow f_H \bar{f}_H$
\item $p p \rightarrow V_H V_H$,
\item $p p \rightarrow T_+ T_+, T_- T_-$
\item $p p \rightarrow T_+ \bar{q}, \bar{T}_+ q, T_+ W^\pm, \bar{T}_+ W^\pm$
\end{enumerate}
% 왜 이 프로세스를 선택했는가. 이 프로세스의 특징은 무엇인가. ... 
These are regarded as the major production channels of LHT particles in the LHC.

Here, we will refer to T-parity odd quarks, $\left\{u_H,~d_H,~c_H,~s_H,~b_H,~t_H\right\}$, as $q_H$; 
heavy gauge bosons, $\left\{ A_H, Z_H, W_H \right\}$, as $V_H$; and
T-parity odd leptons, $\left\{ e_H, \nu_H, \tau_H, \mu_H \right\}$ as $\ell_H$. 
They are produced and decay into a plethora of sets with a variety of possibility. 
To analyse the LHT model, we set benchmark scenarios with several assumptions.

% For T-parity conservation, all of T-parity odd particles are produced in pairs. 
% Among LHT particles, only $T_+$ introduced as a cancellon for Higgs-top loop divergence cancellation
% can be produced in association with SM particles because $T_+$ is T-parity even like SM particles. 
% \note{procduction crosssection plot.}
% Their production cross sections are  monotone decreasing functions, 
% which decrease with $f$ increaseing because
% all of the particles' masses are proportional to $f$.  
%  
% 
% \paragraph{Decay channels}
% \note{Branching ratio} \note{Decay channel table}
% 
% T-odd quark $q_H$ decays into heavy gauge boson $V_H$ and SM quark, 
% if the process is kinematically allowed. 
% % and more into $W_H~q$ than $Z_H~q$ or $A_H~q$.
% % because of electric charge
% Although $V_H$ have several decay channels, under the such condition
% the dominant decay is $A_H$ related ones 
% like $W_H \rightarrow A_H W$, $Z_H \rightarrow A_H H$.  
% If $A_H$ does not decay as the lightest T-parity odd particle,
% the major final state of $q_H$ decay would be SM gauge boson and jets with the missing transverse momentum.
%  
% Regarding of $V_H$ they can also decay to $q_H$ and SM quark 
% under the circumstance such as $M_{V_H} > M_{q_H}$. 
% And $q_H \rightarrow A_H q$ decays are following. 
% The major final state of the decay would be jets and the missing transverse momentum. 
% 
% 
% Thus, the signals to test LHT are various and it is dominantly up to the mass of $q_H$.
% So, We have studied this model with respect to the parameters:
% the center of mass energy $\sqrt{s}$ , the symmetry breaking scale $f$ and 
% the Yukawa coupling parameter of the extended LHT group $\kappa_q$, $\kappa_\ell$
% % The procduction processes have a variety of final state 
% % depending on the kinematical enviorenment and parameter scale. 
% 



\section{Benchmark scenarios}

We distinguish three cases with respect to three model parameters:$f$, $\kappa_q$, and $kappa_\ell$, where $f$ is the symmetry breaking scale of the extended group of LHT;
$\kappa_q$ refers to the Yukawa coupling parameter of T-parity odd quarks $q_H$; and $\kappa_\ell$ is the Yukawa coupling parameter of T-odd leptons $\ell_H$. 
These are the parameters giving rise to mass terms of the LHT particles $q_H, V_H$ and $\ell_H$.
Also, each of them is considered with \emph{conserved T-parity} and with \emph{violated T-parity}. 


\subsection{Fermion universality model}
First, the \fu~model has an assumption that the $\kappa_q$ and $\kappa_\ell$ are equal. 

We have scanned for the parameter space $(f,\kappa)$ where $\kappa~=~\kappa_q~=~\kappa_\ell$. 
Even if the T-odd fermion sectors have the similar mass spectrum, 
the production cross section related colour-charged fermions is much larger than that of the fermions colour neutral. 
The variation of $f$ makes change to the mass of LHT particles, while that of $\kappa$ has effects only on T-odd fermions mass. 
As we can see in the comparison of the \eqs{LHT:VH}{LHT:qH}, if the $\kappa$ is less than $0.4$, 
it becomes true that $M_{V_H} > M_{q_H}$, so $W_H$ and $Z_H$ decays are fermionic. 
Otherwise, $M_{q_H}$ gets heavier than $M_{V_H}$, so we will see $q_H$ decay to $V_H$. 

% \note{FINE TUNNING PROBLEM }; in the region where $f$ is ``small'', there is not so fine-tuned area. 
% I want to talk about that too.

\subsection{Heavy $q_H$ model}
Second, the \hq~model assumes T-parity odd quarks $q_H$ are decoupled in the detectable limits as well as heavier than T-parity odd leptons $\ell_H$,
which results from fixing $\kappa_q = 4.0 $. 
as in the \eqs{LHT:qH}{LHT:lH}, T-parity odd fermions mass terms are proportional to $\kappa_q,~ \kappa_\ell$. 
\begin{align}
 M_{u_H}=\kappa_q f (1+corrections) , M_{d_H}= \kappa_q f \nonumber \\
 M_{\ell_H} = \kappa_\ell f 
\end{align}

With the larger value of $\kappa_q$, $q_H$'s mass is getting much heavier to muli-TeV-scale.
So, $q_H$ is decoupled in the energy level where we looking at and 
it is not possible to detect them in the LHC, which makes we observe effects from other particles which was covered and hiden by $q_H$'s interactions. 
Thus, we can have insight for colour-neutral T-odd particles interactions.

\subsection{Light $\ell_H$ model}
Lastly, for the \lil~model, we fix $\kappa_\ell = 0.2$, which makes $\ell_H$ very light. 
This permits for more particles to decay into lectonic state. 
Especially, heavy gauge bosons $V_H$ are expected to decay into $\ell_H$ and 
it would be interesting to observe how the bounds be changed from that of the \fu~model.


\subsection{R dependence of $T^\pm$}
$T^\pm$'s mass terms \eq{LHT:Tpm} depend on $f$ and $R$, where $R$ is the ratio o the Yukawa coupling parameters $\lam_1/\lam_2$.
We do not scan about $R$ but consider two cases: $R~=~1$ and $R~=~0.2$ where $T^\pm$ is light or heavy so that $T^\pm$ is decoupled. 
When $R$ is equal to 0.2, $T^\pm$ becomes too heavy to be accessed by the LHC experiments.
\more{MAYBE the plot of $T^\pm$ mass for this?}


\section{T-parity violation}
Until now, we only discussed about the case that $A_H$ does not decay because it is the lightest T-odd particle and stable thanks to T-parity conservation. 
As we discussed in \sec{LHT:TPV},
it is possible to introduce the decays of $A_H$ by the gauge anomaly terms. 

As in the \eqs{LHT:TPVww}{LHT:TPVzz}, $A_H$ can decay into $WW$ and $ZZ$. 
Also, taking loop-induced diagrams into account as in \eq{LHT:TPVff}, fermionic decays can appear. 

\note{Feynman DIAGRAMS!!}

Although there are no more invisible particle assumed,
Supersymmetry motivated searches are expected to use for analysing for this case
because there are still notable amount of the transverse missing momentum thanks to 
the leptonic decays $(\ell,\nu)$ of $W$ boson and the invisible decay $(\nu,\nu)$ of $Z$ boson.



\section{CheckMATE}
For testing this model, CheckMATE was used.
CheckMATE is an automatised program to test new physics at the LHC.
From \Feynrules model file, UFO model file, 
it can manage event generating, showering and hadronising by making use of \Madgraph, 
\Pythia and \Delphes. 

It determines the signal


\note{SHOULD rewrite the following}

For the main tasks of this numerical study, we make use of the collider analysis tool \Checkmate{} \cite{??}. 
This program is useful to test a given BSM model in an automatised way. 
It makes again use of the aforementioned  generator \Madgraph to simulate partonic events. 
By making use of the \texttt{UFO} model description file format, \Madgraph is able to 
simulate partonic events for a given BSM model which was implemented in a model building framework
 like \Feynrules or \Sarah. The showering and hadronisation of these events is
 subsequently performed by \Pythia, followed by the fast detector simulation \Delphes
 which considers the effect of measurement uncertainties, finite reconstruction efficiencies 
and the jet clustering of the observed final state objects. 
These detector events are then quantified by various analyses from both \Atlas and \Cms
at centre-of-mass energies of 8 and 13 TeV (more details below). 
Events are categorised in different signal regions and \Checkmate{} determines 
which signal region provides the strongest expected limit. 
If the input model predicts more signal events than are allowed by the observed limit of 
that signal region, \Checkmate{} concludes that the model is \emph{excluded} at the 95\% confidence level, 
otherwise the model is \emph{allowed}. 
For more details on the inner functionality of CheckMATE, we refer to the manual papers in Refs.\cite{??}. 

\more{Credential Limit !}



\section{Final state topologies}

To have comprehensive view of the result plots, 
I need to discuss about topological aspects of LHT particle productions and decays.


\subsection{the production cross sections}
\label{sec:LHTcrosssection}
\begin{figure*}
\centering
\includegraphics[width=0.45\textwidth]{figures/xsbr/samekqkl_fixedk_crosssections13}  \quad
\includegraphics[width=0.45\textwidth]{figures/xsbr/samekqkl_fixedf_crosssections13}
\caption{\more{rewrite this}LHC production cross sections ($\sqrt{s} = \unit[13]{TeV}$) for benchmark models \emph{Fermion Universality/Light $\ell_H$} + \emph{Light $T^\pm$}. 
Left: Dependence on $f$ for fixed $\kappa=1.0$ (solid), $\kappa=2.0$ (dashed). Right: Dependence on $\kappa$ for fixed $f=\unit[1]{TeV}$ (solid), $f=\unit[2]{TeV}$ (dashed).  
Labels in the legend appear in decreasing order of the respective maximum value of the solid lines.}
\label{fig:xs:univ}
% \end{figure*}
% % \begin{figure*}
\includegraphics[width=0.45\textwidth]{figures/xsbr/kq4_fixedk_crosssections13} \quad
\includegraphics[width=0.45\textwidth]{figures/xsbr/kq4_fixedf_crosssections13}
\caption{Same as Fig.~\ref{fig:xs:univ} for benchmark model \emph{Heavy $q_H$} + \emph{Light $T^\pm$}.}\label{fig:xs:kq4}
\end{figure*}
%\note{how the cross section get change w.r.t. the scenarios}\\
We show the production cross section plots with respect to $f$ or $\kappa$ in \figs{xs:univ}{xs:kq4}
where $\sqrt{s}=\unit[13]{TeV}$. 
Since colour neutral fermions $\ell_H$ do not play a crucial role on the production cross section on the LHC, 
changing $\kappa_\ell$ to $0.2$ does not affect to the cross section very much. 
Thus, the production cross section plot of the \fu~model with $\kappa~=~\kappa_q~=~\kappa_\ell$ and that of \lil~model are just alike 
and their cross section plots are shown together. 
Regarding to top partners $T^\pm$, if $T^\pm$ is heavy and in the decoupling limits, $T^\pm$ can hardly afftect on the cross section. 
So, we only show the plots with \emph{light} $T^\pm$ case for cross sections which is experimentally accessible. 
Also, they are only the case of \emph{TPC} because \emph{TPV} cases have rarely effects on the production cross sections.


All the mass of the LHT particles are proportional to $f$, which cause that the cross sections of processes decrease as $f$ increases by kinematical reason. 
Only $q_H$ and $\ell_H$ are affected by $\kappa$ because only their masses depend on $\kappa$. 
$T^\pm$ have dependence on $R$ as well as $f$, but we scanned the processes with fixed $R$ values, 
so we observe $f$ dependence only for $T^\pm$.
And the cross section of the processes related with $T^\pm$ is flat in the $(\kappa, \sigma)$ plot.

% % VH 

Interestingly, even though mass terms of $V_H$ depend only on $f$, we can find a small $\kappa$-dependence from $V_H V_H$ channel of the \fu~model
in the right one of \fig{xs:univ}. This is the effect of $q_H$ in t-channel production that counteract s-channel diagrams. 
% \more{It is because of contributions of $q_H$ in t-channel production of $V_H$ pairs,
% which interfere destructively Drell-Yan-like s-channel vector boson diagrams. }
% % 
This effect is observed where $\kappa~\approx~0.5$ regardless to $f$,
which results in the fact that the production cross section of $V_H$ pair when $\kappa \approx 0.5$ is less than 5 times
in comparison to when there is no effect by this t-channel $q_H$ where $\kappa > 4$ since $q_H$ is decoupled. 
Also by the same reason, in the \fig{xs:kq4} the \hq~model, this effect disappears since $q_H$ are decoupled by construction. 

The production cross section values are roughly $10^3$ fb at most  
and we expect that the LHC can probe the parameter space sensitively in $\sqrt{s} = \unit[13]{TeV}$. 
Furthermore, the cross section is order \unit[$10^{-1}$]{fb} when $f\sim\unit[3]{TeV}$ and
we also can expect detectable event rate which were not the case for LHC Run 1 \mycite{Tonini:2014dza}. 

% From all the cross section plots of the productions in the both plane $(f,\sigma)$ and $(\kappa,\sigma)$, 
From \figs{xs:univ}{xs:kq4} it is clear that none of the processes is significantly high. 
Depending on the values of the points in the parameter space, the cross section relies on very different parameter 
and each process would have a diversity of final states.


When the $\kappa$ is small so that $q_H$ and $\ell_H$ get lighter, 
T-parity odd fermions are expected to be produced more. 
Thanks to the colour factor, the production rate of $q_H$ is 2-3 orders of magnitude higher
than that of $\ell_H$. Thus, $\ell_H$ becomes meaningful only when $f$, $\kappa$ are small 
such as $f \lesssim \unit[1]{TeV}$ and $\kappa \lesssim 0.5$

On the other hand, when $\kappa$ is large, 
the production of heavy vector bosons becomes important 
because their mass is independent of $\kappa$, which is different from that of T-odd fermions
$q_H$ and $\ell_H$ whose mass gets heavier and less produced. 
If $T^\pm$ is light, so it is experimentally accessible case, 
they will be produced as much as $V_H$. 
T-parity even vectorlike top $T^+$ is always negligible 
because $T^+$ is heavier than T-odd vectorlike top $T^-$. 



\subsection{Branching Ratio}
 
%  This discussion is focused on more relevant processes.
We considered only two-body decays and the processes with the branching ration more than $1\%$. 
Mostly, we discuss about the \fu~model and the \lil~model because the \hq~model has very similar decay patterns with the \fu~model.
So, the branching ratio plots about the \hq~model are shown with that of the \fu~model.
$A_H$ only can decay when T-parity is violated 
so, as far as we discuss T-parity conserved case, $A_H$ does not decay at all.

\subsubsection*{Decays of T-parity odd fermions}
\begin{figure*}
\centering
\includegraphics[width=0.45\textwidth]{figures/xsbr/samekqkl_fixedf_BR_for_8880001} \quad
\includegraphics[width=0.45\textwidth]{figures/xsbr/samekqkl_fixedk_BR_for_8880001} \quad
\caption{Branching Ratios of $d_H$ in the \emph{Fermion Universality/Light $\ell_H$} model. 
Items in legend appear in decreasing order of the maximum value of the respective curve. 
Left: Fixed $f = \unit[1]{TeV}$ (solid), $f = \unit[2]{TeV}$ (dashed). Right: Fixed $\kappa = 1$ (solid), $\kappa = 2$ (dashed).}
\label{fig:cm:br1}
\end{figure*}
\begin{figure*}
\centering
\includegraphics[width=0.45\textwidth]{figures/xsbr/samekqkl_fixedf_BR_for_8880002} \quad
\includegraphics[width=0.45\textwidth]{figures/xsbr/samekqkl_fixedk_BR_for_8880002} \quad
\caption{Branching Ratios of $u_H$ in the \emph{Fermion Universality/Light $\ell_H$} model. 
Parameters as in Fig.~\ref{fig:cm:br1}.}
\label{fig:cm:br2}
\end{figure*}


\begin{figure*}
\centering
\includegraphics[width=0.45\textwidth]{figures/xsbr/samekqkl_fixedf_BR_for_8880011} \quad
\includegraphics[width=0.45\textwidth]{figures/xsbr/samekqkl_fixedk_BR_for_8880011} \quad
\caption{Branching Ratios of $e_H$ in the \emph{Fermion Universality} model}
\label{fig:cm:br11}
\end{figure*}

\begin{figure*}
\centering
\includegraphics[width=0.45\textwidth]{figures/xsbr/samekqkl_fixedf_BR_for_8880012} \quad
\includegraphics[width=0.45\textwidth]{figures/xsbr/samekqkl_fixedk_BR_for_8880012} \quad
\caption{Branching Ratios of $\nu_{e,H}$ in the \emph{Fermion Universality} model}
\label{fig:cm:br12}
\end{figure*}
% \paragraph{Decays of $q_H$}
 For T-odd quark $q_H$ branching ratio, the \fu~model and the \lil~model have same aspects in the parameter space as we see in \figs{cm:br1}{cm:br2}
and they are depending on both $f$ and $\kappa$ as so are their mass terms.
Up- and down-type T-odd quarks decay dominantly into $(W_H,~q)$ with the branching ratio $60\%$ in the range that the decays are allowed kinematically.
Second most decay is to $(Z_H,~q)$ with $30\%$ and the last is to $(A_H,~q)$ with $10\%$.
In the right ones of \figs{cm:br1}{cm:br2}, small dependence on $f$ appears because of the factor $v/f$ of each coupling constant, which is various 
depending on the kind of up- and down-type particles. Thus, the dependence that we can see from the curve is also different.

Additionally, because the \fu~model assumes that $\kappa_q~=~\kappa_\ell$, 
the particles related with $\kappa_q$ and $\kappa_\ell$ have the same quantum numbers by construction.
This fact brings about that the decay patterns are identical when they are replaced by the components of $SU(2)$ doublets.
For example, the branching ratio for $\nu_H \to V_H e$ is same with that for $u_H \to V_H d$.

\subsubsection*{Decays of $V_H$} 
\begin{figure*}
\centering
\includegraphics[width=0.45\textwidth]{figures/xsbr/samekqkl_fixedf_BR_for_8880023} \quad
\includegraphics[width=0.45\textwidth]{figures/xsbr/samekqkl_fixedf_BR_for_8880024}
\caption{Branching Ratios of $Z_H$ (left) and $W_H$ (right)  in the \emph{Fermion Universality} (\emph{Heavy $q_H$} is very similar, see text). 
Parameters as in Fig.~\ref{fig:cm:br1}. In both plots, curves corresponding to decays with $\nu, \ell$ or $b$ are nearly identical.}
\label{fig:cm:br2324}
\end{figure*}

\begin{figure*}
\centering
\includegraphics[width=0.45\textwidth]{figures/xsbr/kl2_fixedf_BR_for_8880023} \quad
\includegraphics[width=0.45\textwidth]{figures/xsbr/kl2_fixedf_BR_for_8880024}
\caption{Branching Ratios of $Z_H$ (left) and $W_H$ (right)  in the \emph{Light $\ell_H$} model. $f$ is chosen as in Fig.~\ref{fig:cm:br1}, left.
In both plots, the curves corresponding to decays with $\nu$ or $\ell$ are nearly identical.}
\label{fig:cm:br2324kl}
\end{figure*}
The plot for the branching ratio of the \fu~model is \fig{cm:br2324} and that of the \lil~model is \fig{cm:br2324kl}. 
We show the plots for only the branching ratio with repect to $\kappa$ but not to $f$ because there is no dependence on $f$.  
About the \fu~model, the possible decays of $W_H$ are into $(W, A_H)$, $(q, q_H)$, $(\nu, \ell_H)$, $(\ell, \nu_H)$, $(b, t_H)$ and $(t, b_H)$.
The kinematical condition changes centering around $\kappa \approx 0.5$,
$(W, A_H)$ is the only decay channel in the region of $\kappa > 0.5$ 
while the other states are possible and $W, A_H$ disappears in $\kappa < 0.5$.
Similary, $Z_H$ decays to the same states with that of $W_H$ except for the fact that $h, A_H$ instead of $W, A_H$. 
Again, $Z_H$ decays mainly into $h, A_H$ when it is allowed with $\kappa > 0.5$
whereas this is highly suppressed and other decay channels appear in the region of $\kappa < 0.5 $.

When it comes to the \lil~model, it is different story. 
As $\ell_H$ are holded as light particles, the heavy gauge bosons $W_H$ and $Z_H$ decay into the state 
that consist of $\ell,~\ell_H,~\nu$ and $\nu_H$. 
That is, $W_H \to \nu, \ell_H$, $W_H\to \ell, \nu_H$ with 50\% each 
and $Z_H \to \nu, \nu_H$, $Z_H \to \ell, \ell_H$ also with 50\% each. 
In case of $\kappa_q < 0.5$, decay to $(q, q_H)$ is also allowed which is dominant. 

Note that for the \hq~model, the branching ratios are similar with the \fu~model 
except for the missing of $V_H \to q q_H$. 



\subsubsection*{Decays of $T^\pm$}
\begin{figure}
\centering
\includegraphics[width=0.45\textwidth]{figures/xsbr/samekqkl_fixedk_BR_for_8880008} 
\caption{Branching Ratios of $T^+$ in the \emph{Light $T^\pm$} scenario.}
\label{fig:cm:br7}
\end{figure}
T-odd vectorlike top partner $T^-$ decays only into $A_H$, $t$ with the branching ratio $100\%$. 
The plot \fig{cm:br7} is the branching ratios of $T^+$ in the \lil~model. 
Since $T^+$ is T-parity even particle, it has to decay into a pair of T-odd particles or a pair of SM particles.
There are 4 possibled decays: decay into $(b,W)$ is the prominent channel with the branching ratio $45\%$.
The followings are $(t,Z)$ and $(t,h)$ with $20\%$ each. 
Decaying into a pair of T-odd particles, decay to $(T^-,A_H)$, is also possible although its branching ratio is $\sim 10\%$.
These branching ratios do not depend on $\kappa$ as well as they depend only a little on $f$. 

\subsubsection*{Decays of $A_H$}
\begin{figure}
\centering
\includegraphics[width=0.45\textwidth]{figures/xsbr/tpv_BR_for_8000022}
\caption{Branching Ratios of $A_H$ in the \emph{TPV} benchmark.}
\label{fig:cm:br22}
\end{figure}

The each decay width are followed the calculation of \mycite{Freitas:2008mq}. 
The plot \fig{cm:br22}, its branching ratios are related only with $f$. 
The tendency of the branching ratios are up to the kinematical condition that is about $M_{A_H},~M_W$ and $M_Z$.
As a function of $f$, the mass of $A_H$ is greater than $2\times M_W$ and $2\times M_Z~(\sim \unit[180]{GeV})$ when $f \gtrsim 1200$.
Thus, in the region of $f \gtrsim 1200$ $A_H$ decays mainly into a pair of SM gauge bosons. 

When $f$ goes to the range where $M_{A_H} \lesssim \unit[180]{GeV}$, decays to the pair of SM gauge bosons get suppressed 
but loop-induced leptonic decays play a role. 
If $f$ is smaller than \unit[$900$]{GeV}, decays into the pair of SM fermions become major. 

\subsection{Expected Final topoogies}

\more{comparison with SUSY topologies}