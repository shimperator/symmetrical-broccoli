% %%%%%%%%%%%%%%%%%%%%%%%%%%%%%%%%%%%%%%%%%%%%%%%%%%%%%%%
\chapter{Result in comparison with 13TeV data at the LHC}
\label{chap:LHCresultLHM}
% %%%%%%%%%%%%%%%%%%%%%%%%%%%%%%%%%%%%%%%%%%%%%%%%%%%%%%%
% % 
% 

\section{Collider Results}
% 
% \subsubsection*{General Remarks}
Based on aforementioned discussion, let us grasp the plots of the results from CheckMATE. 
Plots figs.\ref{fig:cmresults:univtpcnotop}-\ref{fig:cmresults:lighttpvtop} are the results for all the scenarios.

Plots have information about the exclusion limits and were drawn by two principles. 
One is to show in which center of mass energy $\sqrt{s}$ the parameter space is excluded between $\unit[8]{TeV}$ and $\unit[13]{TeV}$, which are in the left-hand side row.
The other one is to tell which analyses make the points excluded. 

For the left-hand side ones, the region excluded by $\unit[8]{TeV}$ looks validated by comparison with the result \mycite{Tonini:2014dza}. 
$\unit[13]{TeV}$ extends the excludsion limits thanks to the higher center of mass energy and intergrated luminocities.
Also, these plots contains the mass contours of the particle, which help to find out the excluded points are by which process related which partilcles. 

For the right ones, as just mentioned, plots contain the information about the analyses of the LHC,
which tells that the point is excluded by which analyses. 
Some plots are excluded by one analysis and others are by several ones. 
Furthermore, some points are excluded by several anaylses, in this case the analyses are selected for plotting in the way to choose less number of analyses.



\subsection{\fu~model}
For \fu~model, our discussion proceeds on the base that the Yukawa coupling constants for T-parity odd fermions $\kappa_q$ and $\kappa_\ell$ are equal.
That is, the mass spectrum of T-odd fermions are gathered in the same value.


\subsubsection{T-parity conserved and heavy $T^\pm$ case}

\begin{figure*}[h]
\centering
\includegraphics[width=0.45\textwidth]{figures/limits/kqkl_tpc_813_total_groupallbuttevenodd_andvh} 
\includegraphics[width=0.45\textwidth]{figures/limits/kqkl_tpc_13teV_multian_total_groupallbuttevenodd}
\caption{Results for scenario (\emph{Fermion Universality})$\times$(\emph{Heavy $T^\pm$})$\times$(\emph{TPC})}
\label{fig:cmresults:univtpcnotop}
\end{figure*}

The plot \fig{cmresults:univtpcnotop} is about the result with $R~=~0.2$ which makes $T^\pm$ too heavy to detect from experiments
when T-parity is conserved(\emph{TPC}). 

We can think of the results by the region seperately. 
\paragraph{For $f \gtrsim \unit[1]{TeV} $,} the shape of exclusion limits is parallel to the mass contour of $q_H$. 
The mainly most sensitive analysis for this result is the search for more than 2 jets plus the misisng transverse momentum. 
The decay topology that is the most relevant one is the decay chain of $q_H$.  % into $q~V_H$ follwed by $A_H+X$. 
 Two T-odd quarks $q_H q_H$ decay into $qqA_HA_H~+~X$ is the relevant topology for this analysis. 
And the $A_H$ does not decay and becomes parts of the missing transverse momentum since the T-parity is conserved and $A_H$ is stable.
Thus, the bounds follow the mass contour of $q_H$ which is roughly $f\times \kappa$ where $\kappa=\kappa_q=\kappa_\ell$.
The excluded area is $f\times \kappa < f\times \kappa_{max}$ that $f\times \kappa_{max} \approx \unit[1.5]{TeV}$  when $\sqrt{s}=\unit[8]{TeV}$ 
and $f\times \kappa_{max} \approx \unit[2]{TeV}$  when $\sqrt{s}=\unit[12]{TeV}$. 
% And in this plot, the maximum bound is $f\times \kappa_max$
%because of the $q_H$ particle contributed to this decay.
However, the limit drops a bit stiffer than the $q_H$ mass contour.
This is the effects from the heavy vector bosons production. 
As we discussed in \sec{LHTcrosssection}, there is $q_H$ t-channel productions of $V_H$ and 
this plays a role on this exclusion limit. 


\paragraph{In case of $f \lesssim \unit[1]{TeV}$,} most of area is excluded and there is rarely dependence on $\kappa$ nor $f$
especially in the region $f \lesssim \unit[800]{GeV}$.
This is because T-odd quark is not produced many enough 
 when $\kappa$ is large compared with $f$ in the given energy level ($\sqrt{s}~=~\unit[8]{TeV}$ or $\unit[13]{TeV}$) due to kinematical limits. 
However, the bound depends slightly on $\kappa$ in the region of $f$ between $(800,900)$GeV. 
The reason of this is, as it is mentioned in \sec{LHTcrosssection}, T-odd quarks intefere deconstructively the productions of the heavy vector bosons via t-channel. 
Compared with the region with $f \gtrsim \unit[1]{TeV}$, the main contributing process to the exclusion change,
but the most sensitive analysis does not change, which is the search for the multijets plut the missing transverse momentum. 
The final state topologies come from the hadronic decays of $W$ and $h$ which are from $W_H \to W~A_H$ and $Z_H \to h~A_H$.

\subsubsection{T-parity conserved and light $T^\pm$ case}
\begin{figure*}[h]
\centering
 \includegraphics[width=0.45\textwidth]{figures/limits/kqkl_tpc_813_total} 
\includegraphics[width=0.45\textwidth]{figures/limits/kqkl_tpc_13teV_multian_coverage}
\caption{Results for scenario (\emph{Fermion Universality})$\times$(\emph{Light $T^\pm$})$\times$(\emph{TPC})}
\label{fig:cmresults:univtpctop}
\end{figure*}

To see the sensitivity of the cancellon of top partner $T^\pm$, we choose $R=1.0$ and show the result of it in \fig{cmresults:univtpctop}.
Different from the previous benchmark case, $T^\pm$ is accessible from the experiments 
and the mass of $T^\pm$ is affected only by $f$ since $R$ value is fixed.
As a result, in the large $f$ region, the production of $T^\pm$ get decreased, thus, 
the bound becomes just the same with the result of the previous heavy $T^\pm$ case that we cannot see the effects from the $T^\pm$ production. 
If $T^\pm$ can be produced kinematically as in $f \lesssim 1$TeV, they have notable effects on the bound. 
Not only the $\kappa$-dependece has disappeared in the region of $f \subset (800,900)$GeV, 
but also the vertical exclusion limit is extended up to $f ~\sim~\unit[1.3]{TeV}$. 
In other words, the bound with $\kappa$-dependence is fully dubbed with $T^\pm$ effects.
The most sensitive analysis for this is also the search for multijets with $\ETmiss$. 

% \note{Probbably, the reason why the multijets with $\ETmiss$ is generally most sensitive is 
% because \Checkmate has implemented this analysis with $unit[36.6]{fb^{-1}}$ luminosity 
% not like others with $unit[13.3]{fb^{-1}}$. }

\subsubsection{T-parity violated case}
\begin{figure*}[h]
\centering
\includegraphics[width=0.45\textwidth]{figures/limits/kqkl_tpv_813_total_groupallbuttevenodd} 
\includegraphics[width=0.45\textwidth]{figures/limits/kqkl_tpv_13teV_multian_total_groupallbuttevenodd}
\caption{Results for scenario (\emph{Fermion Universality})$\times$(\emph{Heavy $T^\pm$})$\times$(\emph{TPV})}
\label{fig:cmresults:univtpvnotop}
\end{figure*}


Plots \figs{cmresults:univtpvnotop}{cmresults:univtpvtop} are showing the results of the cases that is the \fu~model with violated T-parity(\emph{TPV}) 
without top partners $T^\pm$ effects(\emph{Heavy $T^\pm$}) and with their effects(\emph{Light $T^\pm$}).  
Again, the plots from the results seems to be seperated to two area: $f \gtrsim \unit[1]{TeV}$ and $f \lesssim \unit[1]{TeV}$.

\paragraph{When $f \gtrsim \unit[1]{TeV}$,} the bound has dependece on $\kappa$ and runs with the contour of $q_H$ mass.
The area of $M_{u_H} \lesssim \unit[2.5]{TeV}$ for $f \approx \unit[1]{TeV}$ and 
$M_{u_H} \lesssim \unit[1.5]{TeV}$ for $f \approx \unit[3]{TeV}$ are excluded.
As T-parity is violated, more final state topologies appear and thus more analyses are contributed to the result.
Especially, the two analyses, \texttt{`atlas\_conf\_2016\_054'} and \texttt{`atals\_conf\_2017\_022'} are almost equally sensitive to the exclusion area.
The former one is the search for 1 lepton, multijets and $\ETmiss$ and the latter one is for no lepton, multijets and $\ETmiss$.
The fact that they are sensitive to the result is understandable if we take into account the final state of SM gauge boson decays. 
From the decays of SM gauge bosons, the final states that did not exist from the previous benchmark scenarios appear,
which are produced via leptonic and hadronic decay of $W$ and $Z$ bosons. 
By each signal region of the two searches, the similar number of events are observed. 
% \more{Overall, gain the similar event rate in the respective signal regions of these two studies. ? }
% % % %  final state -  lepton, why 2016-054 is more sensitive
In general, $A_H$ decays to $WW$ more than to $ZZ$, in addition when $f$ is smaller, it becomes more ovbious. 
Since the decay of $W$ produces more charged leptons than that of $Z$, we can make a guess that there are more leptonic final state in the region with smaller $f$. 
And this makes the analysis with lepton in the final states more sensitive and 
we observe it from the plot \fig{cmresults:univtpvnotop} in $f \subset (1,1.5)$TeV. 

Another interesting thing is that the exclusion area does not change dramatically, which is because of the role of neutrinos in the final states. 
From the processes that we are observing, basically a pair of heavy photons $A_H$ is produced from $q_H q_H$ or $V_H V_H$ and
this decays into $WW$ or $ZZ$ with broken T-parity. 
As $W$ and $Z$ decay, the missing transverse momentum originated from neutrinos is observed. 
Assuming that gauge bosons decay into $\nu+X$ just averagely $25\%$ since we have 4 of gauge bosons from a pair of heavy photons, 
so, the missing transverse momentum appears with at least $70\%$ of $\ETmiss$ of \emph{TPC} case.
% Although the amount of $\ETmiss$ gets decreased, the bound does not change significantly thanks  %even though the T-parity is violated.
The final states are the same visible state with the previous case and additional boosted particles, which may help the event pass the cut better. 
This is why the bound does not change significantly compared with the previous case.


\paragraph{Next, for $f \lesssim \unit[1]{TeV}$,} let us discuss rather $\kappa$ independent limit.  
Compared with the \fu~model with conserved T-parity, the vertical limit is extended. 
The parameter space is excluded when $f \lesssim \unit[1]{TeV}$ with $\kappa \lesssim \unit[1.5]{TeV}$ 
and $f \lesssim \unit[1.1]{TeV}$ with $\kappa \lesssim \unit[4.0]{TeV}$. 
To figure out the reason of the exclusion, we again scrutinize the analyses of the LHC from the right one of the plot \fig{cmresults:univtpvnotop}. 
The multijets search that used to have a crucial impact on the bound for the previous cases is not the most important one any more.
For this result, preferably the electroweakino search plays more important role. Specifically, the signal region \texttt{SR-slep-e} generates the exclusion bound.
This analysis looks for three high-$p_T$ cahrged lepton not coming from a pair of $W$ or $Z$ and a large amount of $\ETmiss$
It was not possible to pass this cut for the case with conseved T-parity because the decay chain without $A_H$ decays, for example,
$q_H q_H \to qq W_H W_H \to qqWWA_H A_H$ have only two leptons. As decaying $A_H$ with violated T-parity, third charged lepton can appears. 
Furthermore, there is no constraints about jet multiplicity for the final states, which may bring about a large number of events expected via this analysis. 



\begin{figure*}[h]
\centering
\includegraphics[width=0.45\textwidth]{figures/limits/kqkl_tpv_813_total} 
\includegraphics[width=0.45\textwidth]{figures/limits/kqkl_tpv_13teV_multian_coverage}
\caption{Results for scenario (\emph{Fermion Universality})$\times$(\emph{Light $T^\pm$})$\times$(\emph{TPV})}
\label{fig:cmresults:univtpvtop}
\end{figure*}
\paragraph{The top partner $T^\pm$ is light} and accessible by experiments, which extend the exclusion limit by $\unit[100]{GeV}$, as we can see from the plot \fig{cmresults:univtpvtop}. 
So the result differece generated by light $T^\pm$ is not so much, which is different from the case of \emph{TPC}. 
And still, the electroweakino search is the most important analysis. 
For $T^\pm$ search, the final topologies produced from $p p \to T^- \bar{T^-} \to (bW)(bW) WW VV$ have crucial roles. 
The more events come from it, the more exclusion area is extended.
Of coures, since $T^\pm$ depends on $R$ and $f$, the limits generated by $T^\pm$ very much depends on the value of $R$. 
We just set one case with $R=1$ to see how the results would be. 


\paragraph{The multijet analysis,} \texttt{`atals\_conf\_2017\_022'}, does not have so much impact on the results not like the \emph{TPC} case.
This analysis covers various hierarchy and topologies of Supersymmetric particles $\tilde{g},~\tilde{q}$.
And it is targetting of r

\more{Supersymmetric Standard model search related $q_H$  behaviour}\\
\more{$T^+ \to t A_H$ topology}\\



\subsection{\hq~model}

\begin{figure*}
\centering
\includegraphics[width=0.45\textwidth]{figures/limits/kq4_tpc_813_total_groupallbuttevenodd_andvh} 
\includegraphics[width=0.45\textwidth]{figures/limits/kq4_tpc_13teV_multian_total_groupallbuttevenodd}
\caption{Results for scenario (\emph{Heavy $q_H$})$\times$(\emph{Heavy $T^\pm$})$\times$(\emph{TPC})}
\label{fig:cmresults:heavytpcnotop}
\end{figure*}

\begin{figure*}
\centering
\includegraphics[width=0.45\textwidth]{figures/limits/kq4_tpc_813_total} 
\includegraphics[width=0.45\textwidth]{figures/limits/kq4_tpc_13teV_multian_coverage}
\caption{Results for scenario (\emph{Heavy $q_H$})$\times$(\emph{Light $T^\pm$})$\times$(\emph{TPC})}
\label{fig:cmresults:heavytpctop}
\end{figure*}

\begin{figure*}
\includegraphics[width=0.45\textwidth]{figures/limits/kq4_tpv_813_total_groupallbuttevenodd} 
\includegraphics[width=0.45\textwidth]{figures/limits/kq4_tpv_13teV_multian_total_groupallbuttevenodd}
\caption{Results for scenario (\emph{Heavy $q_H$})$\times$(\emph{Heavy $T^\pm$})$\times$(\emph{TPV})}
\label{fig:cmresults:heavytpvnotop}
\end{figure*}

\begin{figure*}
\centering
\includegraphics[width=0.45\textwidth]{figures/limits/kq4_tpv_813_total} 
\includegraphics[width=0.45\textwidth]{figures/limits/kq4_tpv_13teV_multian_coverage}
\caption{Results for scenario (\emph{Heavy $q_H$})$\times$(\emph{Light $T^\pm$})$\times$(\emph{TPV})}
\label{fig:cmresults:heavytpvtop}
\end{figure*}

We discuss of the results with the \hq~model that $\kappa_q$ is fixed value as $4.0$. 
The result plots are figs.\ref{fig:cmresults:heavytpcnotop}-\ref{fig:cmresults:heavytpvtop}. 
Fixed large value of $\kappa_q$ causes that $q_H$ is decoupled in the energy scale of experiments 
and the production channels in association with $q_H$ get irrelevant to the exclusion limits.
For this reason, there are $\ell_H$ mass contours on the plots instead of that of $q_H$. 

Since $q_H$ become decoupled with the condition of the \hq~model, the topology that mainly contributes to the results is $\ell_H$ decay into $\ell~V_H$ instead of $q_H~\to~q~V_H$. 
That is, the final dominant topologies change to multilepton state from multijets states. 
In the hadron collider LHC, the production event rate of $\ell_H \ell_H$ is less by 2 to 3 orders of magnitude than that of $q_H q_H$,
it is expected that the region where $q_H q_H$ has crucial on would get weaker, for example, in large $f$ 
and small $\kappa$.

Besides, while the $V_H V_H$ production is affected by the $\kappa_q$ from t-channel $q_H$, 
$\kappa_\ell$ does not have any effect on it. Thus, the exclusion area formed by the contrubution of $V_H V_H$ production channel is predicted to have the result 
with $\kappa~=~4.0$ of the \fu~model. 

As we saw in the previous benchmark, the results are diveded in two area. 
\paragraph{For $f$ is similar with $\unit[1]{TeV}$,} the most sensitive processes are ones related with heavy vector bosons $V_H$ production or cancellon of top partners $T^\pm$ productions.
Both are irrelevant to $\kappa_\ell$, thus the exclusion limit becoms vertical line. There are more or less around $\unit[1]{TeV}$ area. 
 \begin{table}[h]
\begin{center}
  \begin{tabular}{ l | c | c }
    \hline
     & $T^\pm$ is heavy &  $T^\pm$ is light \\ \hline
    T-parity is conserved(\emph{TPC}) & $f~\gtrsim~\unit[950]{GeV}$ & $f~\gtrsim~\unit[1350]{GeV}$ \\ \hline
    T-parity is broken(\emph{TPV}) & $f~\gtrsim~\unit[1100]{GeV}$ & $f~\gtrsim~\unit[1200]{GeV}$ \\
    \hline
  \end{tabular}
\end{center}
 \end{table}

This table shows the allowed area with respect to the cases and they are in agreement with the results with $\kappa~=~4.0$ of the \fu~model
, which tells us that the effect of $\kappa_\ell$ is tiny. 
Also, the most influential topologies are still multijets final states for the T-parity conserved case. 
On the other hand, multilepton final states get to have more effects on the results for the T-parity violated case.


\paragraph{For $\kappa_\ell$ is less than $0.5$,} the important feature is that the mass hierarchy between heavy vector $V_H$ and T-odd lepton $\ell_H$ are changed.
Boosted leptonic states are produce by this decay, which is fitted to the multilepton analysis and this excluded $f$ upto $\unit[1.9]{TeV}$.
Of course, these decays depend on $\kappa_\ell$ and we can check it from the bound.

\paragraph{The bound of $f\times \kappa_{max}$} has vanished as $q_H$ dissolved in the detectable energy scale by $\kappa_q=4.0$.
In case of \emph{TPC}, this is also because of the fact that $\ell_H \ell_H$ has too less production event rate to give any exclusion limit from this even for $\sqrt{s}~=~\unit[13]{TeV}$.
But in \emph{TPV} case, there is the exclusion bound along with the mass contour of $\ell_H$. It is thanks to the increase of leptonic decays from gauge bosons. 
\bigskip

From the \hq~model, we could find out that $q_H$ has crucial impacts on the exclusion limits in the region $\kappa \lesssim 1.5$ at the LHC.
But the exclusion limits contributed by heavy gauge bosons which is almost vertical is strong regardless of the choice of $\kappa$ for T-odd fermion sectors.
 
 

\subsection{\lil~model}
\begin{figure*}
\centering
\includegraphics[width=0.45\textwidth]{figures/limits/kl2_tpc_813_total_groupallbuttevenodd_andvh} 
\includegraphics[width=0.45\textwidth]{figures/limits/kl2_tpc_13teV_multian_total_groupallbuttevenodd}
\caption{Results for scenario (\emph{Light $\ell_H$})$\times$(\emph{Heavy $T^\pm$})$\times$(\emph{TPC})}
\label{fig:cmresults:lighttpcnotop}
\end{figure*}

\begin{figure*}
\includegraphics[width=0.45\textwidth]{figures/limits/kl2_tpc_813_total} 
\includegraphics[width=0.45\textwidth]{figures/limits/kl2_tpc_13teV_multian_coverage}
\caption{Results for scenario (\emph{Light $\ell_H$})$\times$(\emph{Light $T^\pm$})$\times$(\emph{TPC})}
\label{fig:cmresults:lighttpctop}
\end{figure*}

\begin{figure*}
\centering
\includegraphics[width=0.45\textwidth]{figures/limits/kl2_tpv_813_total_groupallbuttevenodd} 
\includegraphics[width=0.45\textwidth]{figures/limits/kl2_tpv_13teV_multian_total_groupallbuttevenodd}
\caption{Results for scenario (\emph{Light $\ell_H$})$\times$(\emph{Heavy $T^\pm$})$\times$(\emph{TPV})}
\label{fig:cmresults:lighttpvnotop}
\end{figure*}

\begin{figure*}
\includegraphics[width=0.45\textwidth]{figures/limits/kl2_tpv_813_total} 
\includegraphics[width=0.45\textwidth]{figures/limits/kl2_tpv_13teV_multian_coverage}
\caption{Results for scenario (\emph{Light $\ell_H$})$\times$(\emph{Light $T^\pm$})$\times$(\emph{TPV})}
\label{fig:cmresults:lighttpvtop}
\end{figure*}
For the \lil~model, we fix $\kappa_\ell$ as $0.2$ so that T-parity odd leptons become even lighter and the mass gap between T-odd quarks and leptons gets broadened. 


\more{in \fu~model, while mainly $Z_H \to h Z_H$ and $W_H \to W A_H$, \\
instead, in \lil~model, $Z_H \to \ell \ell_H$ and $WH \to \ell_H \nu$($\nu_H \ell$) 
this leptonic decay follwed by $\ell_H \to \ell A_H$ or $\nu_H \to \nu A_H$}

\paragraph{two area}
\more{As previous benchmarks, main exclude region $f \lesssim \unit[1.6]{TeV}$ 
and $\kappa_q \lesssim 1.2$ }


\paragraph{For smaller $\kappa_q$ }
\more{and larger $f$, the bound looks similar with that of the \fu~model }

\more{CheckMATE said it is sensitive to the search for multileptons \emph{analysis-name-here}}

\more{from $\kappa_q <0.5$, $V_H$s have dominantly hadronic decays} and generate multijet final state.
This repeat the similar exclusion limit for smaller $\kappa_q$ value with the \fu~model.

\more{Since $q_H$ related final states are irrelevant to $T^\pm$ sector, the results does not change 
depending on the presence of $T^\pm$ }

\more{Furthermore,  in case of \emph{TPV}, the limits get slightly weakened because }
the fact that $A_H$ decays results in the less $\ETmiss$ cut efficiency (????)

\paragraph{For larger value of $\kappa_q$,} again, the exclusion limit forms almost vertical area.
The bound is almost vertical because the heavy gauge bosons $V_H$ that have impacts on each point of this exclusion area
are given effects from $f$ not from $\kappa_q$.
However, as mentioned earlier, the production of heavy gauge bosons earns a little bit of contributions 
from $t$-channel $q_H$, which causes a little bit of dependence on $\kappa_q$.

By comparison to the result of the \fu~model, we observe that the exclusion area is extended considerably,
which is excluded at $f \lesssim \unit[1.6]{TeV}$ for $\kappa_q \lesssim 1.5$ and $f \lesssim \unit[2]{TeV}$ for $\kappa_q < 5.0$.
The reason why the bound is enlarged pretty much is thanks to the clean signal of boosted leptons.
As the analyses coverage map on the plots says, this is excluded by the multilepton analysis that
detect the boosted leptons stemming from the heavy gauge bosons not from SM gauge bosons $W$ and $Z$,
so the signal of the process is clean and distinguishing with less SM background. 


\paragraph{The heavy top partner $T^\pm$} does not generate any noticeable because the bound by the processes related with $V_H$ is too strong.
The production of $T^\pm$ affects only for $f\lesssim\unit[1.35]{TeV}$ which is already covered by $V_H$ processes.
Also, the final states of $T^\pm$ decay does not overlap with that of $V_H$, hence there is no enhancement as well.

\paragraph{For T-parity violated case,} there is no extension from the result when $A_H$ does not decay,
even the bound gets a bit weaker and this is because decaying of $A_H$ give rise to decrease of $\ETmiss$ cut efficiency.
\more{dlrj antmsakfdlswl wkf ahfmrpTdma.}


\paragraph{Overall,} from the \lil~model setting, the dominant signal of the heavy gauge boson $V_H$ decay has changed 
and this extends the exclusion area notably. 
The purpose of setting $\kappa_\ell$ as certain value and scanning $(f,\kappa_q)$ on the parameter space is 
to see how the result change depending on the parameter $\kappa_\ell$ as well as $(f, \kappa_q)$.
So, let us discuss the tendency of the result in the \lil model compared with that of the \fu~model.

With $\kappa_\ell$, the heavy gauge boson $V_H$ decays into $\ell_H \ell$ with 100\% and of course,
its decay width depends on the mass of $\ell_H$. The heavier the $\ell_H$ mass is, the less $V_H$ decay into it 
and as getting heavier $\ell_H$, the result plots of the \lil~model will recover to that of the \fu~model.

\more{$V_H\to \ell \ell' A_H$ decay kinetic configuration.}

different mass will cause different energy distribution? 
Since the mass gap between $M_{V_H}$ and $M_{A_H}$ is around $\unit[750]{GeV}$ at $f \sim \unit[1.5]{TeV}$,
the final state lepton will be so energetic that the signal pass the constraints?


\subsection{14TeV projections}
\begin{figure*}
\centering
\includegraphics[width=0.45\textwidth]{figures/limits/kqkl_tpc_1314_total_groupallbuttevenodd_andvh} 
\includegraphics[width=0.45\textwidth]{figures/limits/kqkl_tpc_14teV_multian_total_groupallbuttevenodd}
\caption{Expected results at $\sqrt{s}=\unit[14]{TeV}, \int \mathcal{L} = \unit[3000]{fb}^{-1}$ for scenario (\emph{Fermion Universality})$\times$(\emph{Heavy $T^\pm$})$\times$(\emph{TPC}). }
\label{fig:cmresults:14tev:samekqkl}
\includegraphics[width=0.45\textwidth]{figures/limits/kq4_tpc_1314_total_groupallbuttevenodd_andvh} 
\includegraphics[width=0.45\textwidth]{figures/limits/kq4_tpc_14teV_multian_total_groupallbuttevenodd}
\caption{Expected results at $\sqrt{s}=\unit[14]{TeV}, \int \mathcal{L} = \unit[3000]{fb}^{-1}$ for scenario (\emph{Heavy $q_H$})$\times$(\emph{Heavy $T^\pm$})$\times$(\emph{TPC}).}
\label{fig:cmresults:14tev:kq4}
\includegraphics[width=0.45\textwidth]{figures/limits/kl2_tpc_1314_total_groupallbuttevenodd_andvh} 
\includegraphics[width=0.45\textwidth]{figures/limits/kl2_tpc_14teV_multian_total_groupallbuttevenodd}
\caption{Expected results at $\sqrt{s}=\unit[14]{TeV}, \int \mathcal{L} = \unit[3000]{fb}^{-1}$ for scenario (\emph{Light $\ell_H$})$\times$(\emph{Heavy $T^\pm$})$\times$(\emph{TPC}).}
\label{fig:cmresults:14tev:kl2}
\end{figure*}
